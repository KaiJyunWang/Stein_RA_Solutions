The following are some auxiliary results used in the proofs 
of the exercises. Most of them are standard results in 
analysis while have not been encountered in my previous 
studies. In order to enhance the completeness of the 
solutions, I have included them in this appendix.
\begin{theorem}[Baire Category Theorem]\thmlab{A.1}
    Let $X$ be a complete metric space. Then the 
    intersection of countably many dense open sets in $X$ 
    is dense in $X$.
\end{theorem}
\begin{pf}
    Let $G_n$ be dense open sets in $X$. We first claim 
    that $\not\exist$ perfect $F$ such that $G_n^c$ is 
    dense in $F$. Suppose not, then we can pick $x\in 
    G_n\bigcap F$ such that $\forany B_\epsilon(x)$, 
    $B_\epsilon(x)\bigcap G_n^c\neq\varnothing$ since 
    $G_n^c$ is dense in $F$. But $B_\epsilon(x)\subset
    G_n$ for some $\epsilon$, posing a contradiction.

    Now pick some perfect $E$. Since $G_n^c$ is not dense 
    in $E$, $\exist x_1\in G_1\bigcap E$. Then $\exist
    \epsilon_1>0$ such that 
    $\overline{B_{\epsilon_1}(x_1)}\subset G_1$ and 
    $\overline{B_{\epsilon_1}(x_1)}$ is perfect. Since 
    $G_2^c$ is not dense in $E$, $\exist x_2\in G_2\bigcap
    \overline{B_{\epsilon_1}(x_1)}$. Then $\exist
    \epsilon_2 = \min\brc{\epsilon_1/2,\delta_2}$, where 
    $\delta_2$ is a small number such that 
    $\overline{B_{\delta_2}}\subset G_2$. The procedure 
    continues and we obtain a sequence that 
    $\overline{B_{\epsilon_1}(x_1)}\supset
    \overline{B_{\epsilon_2}(x_2)}\supset\cdots$, which 
    are closed, bounded and nonempty with 
    $\lim_{n}\diam{\overline{B_{\epsilon_n}(x_n)}}
    \leq \lim_{n} 2\epsilon_1 2^{-n+1} = 0$. 

    We now make the second claim: For any sequence of 
    closed and bounded sets in $X$ with 
    $E_n\supset E_{n+1}$ and $\lim_{n}\diam{E_n}=0$, 
    $\bigcap_n E_n$ consists of exactly one point. To see 
    this, suppose $x\neq y$ are distinct points in 
    $\bigcap_n E_n = \lim_{n}E_n$. Since $x$ and $y$ are 
    distinct, $d(x,y)>0$ and $\diam{E_n} = 
    \sup_{x',y'\in E_n} d(x',y')\geq d(x,y)$. Let 
    $n\to\infty$, then $\lim_{n}\diam{E_n}\geq d(x,y)>0$, 
    which is a contradiction.

    By the second claim, $\bigcap_n G_n\supset
    \bigcap_n\overline{B_{\epsilon_n}(x_n)}
    \neq\varnothing$. Futhermore, the choice of $x_1$ is 
    dense in $X$. This furnishes the proof.
\end{pf}

\begin{definition*}
    A topological space $(X,\mathcal{T})$ is said to be 
    \textbf{normal} if for every pair of disjoint closed 
    sets $C$ and $D$ in $X$, there exist disjoint open sets 
    $U$ and $V$ such that $C\subset U$ and $D\subset V$.
\end{definition*}

\begin{lemma}[Urysohn's Lemma]\lemlab{A.2}
    A topological space $(X,\mathcal{T})$ is normal if 
    and only if for every pair of nonempty disjoint closed 
    sets $C$ and $D$ in $X$, there is a continuous function 
    $f:X\to[0,1]$ such that $f(C)=\set{0}$ and 
    $f(D)=\set{1}$.
\end{lemma}
\begin{pf}
    We shall first prove the necessity. Let $C$, $D$ be 
    disjoint closed sets in $X$ and $f$ be the function as 
    in the statement of the lemma. Then $C\subset 
    f^{-1}([0,1/2))$ and $D\subset f^{-1}((1/2,1])$. Since 
    $f$ is continuous, the preimages are open. This proves
    the necessity.

    The sufficiency is more complicated. We shall first 
    construct a collection of open sets $\Set{U_q}{q\in Q}$ 
    satisfying that for any $p<q$, $\overline{U_p}\subset
    U_q$, where $Q = \Q\bigcap[0,1]$. 

    To begin, let $U_1 = X-D$ and by the normality of $X$, 
    pick an open $U_0$ such that $C\subset U_0\subset 
    \overline{U_0}\subset U_1$. Note that the desired property 
    holds. Now suppose that $U_p$ has been constructed for 
    $p\in\set{p_1,\ldots,p_n} = P_n$, we are now to construct 
    $U_{p_n+1}$ for $p_{n+1}$. Since $P_n$ is finite, we may 
    find $q,r\in Q$ such that $p_{n+1}$ is the only numbers in 
    $P_n$ that $q<p_{n+1}<r$. By the inductive hypothesis, we 
    have $\overline{U_q}\subset U_r$. Again, by the normality, 
    we may choose an open $U_{p_{n+1}}$ such that 
    $\overline{U_q}\subset U_{p_{n+1}}\subset 
    \overline{U_{p_{n+1}}}\subset U_r$. With this, the 
    collection of open sets still preserves the desired 
    property that $p<q$ implies $\overline{U_p}\subset U_q$. 
    As the process proceeds, we obtain a collection of open 
    sets $\Set{U_q}{q\in Q}$ preserving the desired property. 

    Next, we extend the index set of our collection to $\Q$ 
    by defining $U_q = \varnothing$ if $q<0$ in $\Q$ and 
    $U_q = X$ if $q>1$. With the extension, we still have the 
    desired property that $p<q$ implies $\overline{U_p}\subset
    U_q$ for $\Set{U_q}{q\in\Q}$.

    For each $x\in X$, we define $\Q(x) = 
    \Set{q\in\Q}{x\in U_q}$. Observe that $\Q(x)$ is nonempty 
    since $x\in U_q=X$ for any $q>1$ and also $\Q(x)$ is 
    bounded below by $0$ since if $q<0$, then $x\not\in U_q 
    = \varnothing$. Remark that for each $x\in X$, $\Q(x)$ 
    contains every rational numbers greater than $1$ and some 
    lies between $0$ and $1$.

    By the axiom of completeness, we can define the function 
    $f:X\to[0,1]$ such that $f(x) = \inf\Q(x)$. Such function 
    is well-defined since $\Q(x)$ is bounded below by $0$ and 
    above by $1$. It remains to show that $f$ is continuous 
    and seperates $C$ and $D$.

    The first claim is that $f(C)=\set{0}$ and $f(D)=\set{1}$.
    Indeed, if $x\in C$, then $x\in U_0\subset U_p$ for all 
    $p>0$. Thus, $\inf\Q(x)\leq 0$. This gives $f(x)=0$. If 
    $x\in D$, then $x\not\in U_1$ and hence $\inf\Q(x)\geq 1$. 
    Then $f(x)=1$. Hence we conclude that $f(C)=\set{0}$ and 
    $f(D) = \set{1}$. The claim further implies that $f$ 
    separates $C$ and $D$. 

    Before claiming the continuity of $f$, we shall observe 
    that for any $x\in\overline{U_q}$, $f(x)\leq q$ by the 
    construction that $\overline{U_q}\subset U_r$ for any 
    $r>0$. Also, for any $x\not\in U_q$, $f(x)\geq q$ since 
    $x\not\in U_p$ for any $p<q$. With these observations, 
    we are now ready to show the continuity of $f$. 

    For any open interval $U = (a,b)$ that lies in $[0,1]$, 
    we need to show that $f^{-1}(U)$ is open in $X$. Fixing 
    any $x\in f^{-1}(U)$, we can find rational numbers $p,q$ 
    such that $a<p<f(x)<q<b$. $p<f(x)$ implies that $x\not\in
    \overline{U_p}$; otherwise, $f(x)\geq p$, forming a 
    contradiction. By similar argument, $x\in U_q$. As a 
    result, $x\in U_q-\overline{U_p}$. Let the open set be 
    $V = U_q-\overline{U_p}$. 

    To show that $f(V)\subset U$, let $y\in V$. Then $y\in 
    U_q\subset\overline{U_q}$ and $f(y)\leq q<b$. Also, $y\not
    \in\overline{U_p}\supset U_p$ and hence $f(y)\geq p>a$. 
    Thus $f(y)\in[p,q]\subset U$. Therefore $f$ is indeed 
    continuous. The proof is now complete.
\end{pf}

\begin{definition}
    A topological space $X$ is said to be a 
    \textbf{Hausdorff space} if for any pair of distinct
    points $x$ and $y$ in $X$, there exist disjoint open 
    sets $U$ and $V$ such that $x\in U$ and $y\in V$. 
\end{definition}

\begin{lemma}\lemlab{A.3}
    Let $A$ and $Y$ be closed subspaces of a normal Hausdorff 
    space $X$ and let $U$ be an open neighbourhood of $Y$ in 
    $X$. Assume that $C\subset A$ is a closed neighbourhood 
    in $A$ of $Y-A$, contained in $U-A$. Then there exists a 
    closed neighbourhood $Z$ of $Y$, contained in $U$, such 
    that $Z-A$ equals $C$.\footnote{The proof of this lemma 
    and the Tietze Extension Theorem are taken from 
    \url{https://link.springer.com/content/pdf/10.1007/s000
    130050272.pdf}.}
\end{lemma}
\begin{pf}
    The case $A = \varnothing$ is elementary and well known; 
    applying it to the situation at hand we obtain first a 
    closed neighbourhood $Z'''$ of $Y$ contained in $U$. It 
    will suffice to construct (from $Z'''$) a closed 
    neighbourhood $Z'$ of $Y$ contained in $U$ such that
    $Z'\bigcap A$ is contained in $C$, since then $Z\coloneqq 
    Z'\bigcup C$ will have all the properties stated.

    Let $D$ be the closure of $A-C$. Since $C$ is assumed to 
    be a neighbourhood of $Y\bigcap A$ in $A$, the closed sets 
    $Y$ and $D$ are disjoint. Hence $Y$ has a closed 
    neighbourhood $Z''$ disjoint from $D$. This implies 
    $Z''\bigcap A\subset C$, and $Z'\coloneqq Z''\bigcap Z'''$ 
    is a closed neighbourhood of $Y$ as required.
\end{pf}

\begin{theorem}[Tietze Extension Theorem]\thmlab{A.4}
    Let $X$ be a normal Hausdorff space and $A$ be a 
    closed subset of $X$. If $f:A\to\R$ is a continuous 
    function, then $f$ can be extended to a continuous 
    function $F:X\to\R$. That is, $F|_A = f$. 
\end{theorem}
\begin{pf}
    For $r\in[0,1]$, let $A(r) = \set{x\in A\mid f(x)\leq r}$. 
    Put $B = \Set{r\in[0, 1]}{2^nr\text{ is integral for some }n}$.
    For $r \in B$ we construct, by induction on the exponent of 
    $2$ in the denominator of $r$, closed subsets $X(r)\subset X$ 
    such that the following two conditions hold:
    \begin{thmenum}
        \item $X(r) \cap A = A(r)$,
        \item $X(s)$ is a neighbourhood of $X(r)$ if $r < s$.
    \end{thmenum}
    We may take $X(0)=A(0) $ and $X(1)=X$. Assume that $X(r)$ 
    with the properties above have been constructed for all 
    $r\in(0,1)$ such that $2^nr$ is an integral. Since 
    $X(2^{-n}(i+1))$ is a neighborhood of $ X(2^{-n}i)$, 
    there is, by the above lemma, a closed neighborhood 
    $X(2^{-n-1}(2i+1))$ of $ X(2^{-n}i) $ within the interior 
    of $X(2^{-n}(i+1))$, having the prescribed intersection 
    $X(2^{-n-1}(2i+1)) \cap A = A(2^{-n-1}2i + 1)$.

    The rest of the proof follows the classical Urysohn argument
    that we presented in \lemref{A.2}:
    \[
        g(x) := \inf \Set{r\in B}{x\in X(r)}
    \]
    defines a continuous function on $X$, extending $f$. 
    Continuity follows simply from the fact that, for 
    $r,s\in B$ with $r<s$, this function has values greater 
    than $r$ and less than $s$ on the open set 
    $X^\circ(s)-X(r)$. 

    Finally, we may decompose $f$ into $g: X\to (0,1)$ and a 
    homeomorphism $h: (0,1)\to\R$. Our previous result shows 
    that $\exist$ continuous extension $G:X\to[0,1]$. Since 
    $G^{-1}(\set{0,1})$ is closed and disjoint from $A$, by 
    \lemref{A.2}, the proof is finished.
\end{pf}

\begin{lemma}[Young's Inequality]\lemlab{A.5}
    Let $1\leq p,p'\leq\infty$ with $\frac{1}{p}+\frac{1}{p'}=1$. 
    Then for all $a,b\geq 0$, 
    \begin{equation*}
        ab \leq \frac{a^p}{p} + \frac{b^{p'}}{p'}.
    \end{equation*}
    Moreover, the equality holds if and only if $a^p = b^{p'}$.
\end{lemma}
\begin{pf}
    If $a = 0$ or $b = 0$, the inequality is trivial. 
    Assume that $a,b>0$. By the concavity of logarithm, 
    \begin{equation*}
        \log\pth{\frac{a^p}{p} + \frac{b^{p'}}{p'}} \geq \frac{1}{p}\log a^p + \frac{1}{p'}\log b^{p'} = \log ab.
    \end{equation*}
    Exponentiating both sides yields the desired inequality.
\end{pf}

\begin{theorem}[H\"older's Inequality in $\L^p$]\thmlab{A.6}
    Let $1\leq p,p'\leq\infty$ with $\frac{1}{p}+\frac{1}{p'}=1$. 
    Then for all $f\in\L^p$ and $g\in\L^{p'}$, 
    \begin{equation*}
        \norm{fg}_1 \leq \norm{f}_p\norm{g}_{p'}.
    \end{equation*}
    Moreover, the equality holds if and only if $f = cg$ for some 
    constant $c$.
\end{theorem}
\begin{pf}
    For the case $p = 1$ and $p' = \infty$, notice that
    \begin{equation*}
        \abs{fg} \leq \abs{f}\esssup \abs{g} \implies 
        \norm{fg}_1 = \int \abs{fg}d\mu \leq \int \abs{f}\esssup\abs{g}d\mu = \norm{f}_1\norm{g}_\infty.
    \end{equation*}
    For the case $p = \infty$ and $p' = 1$, the proof is similar. 
    Now suppose $1<p<\infty$ and $1<p'<\infty$. If one of $f$ or 
    $g$ is zero, the inequality is trivial. Without loss of 
    generality, we may assume that $\norm{f}_p = \norm{g}_{p'} = 1$. 
    By the \lemref{A.5}, 
    \begin{equation*}
        \abs{fg} \leq \frac{\abs{f}^p}{p} + \frac{\abs{g}^{p'}}{p'}.
    \end{equation*}
    Integrating both sides yields
    \begin{equation*}
        \norm{fg}_1 = \int \abs{fg}d\mu \leq \int \frac{\abs{f}^p}{p}d\mu + \int \frac{\abs{g}^{p'}}{p'}d\mu 
        = \frac{1}{p} + \frac{1}{p'} = 1.
    \end{equation*}
    Hence we obtain the desired inequality. The equality holds if 
    and only if $\abs{f}^p = \abs{g}^{p'}$ a.e.\ by the \lemref{A.5}. 
    In general, the equality holds if and only if $f = cg$ a.e.\ for some 
    constant $c$ after scaling the both sides of the inequality by $c$.
\end{pf}