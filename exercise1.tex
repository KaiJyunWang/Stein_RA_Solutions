\begin{exercise}\exelab{1.1}
    Prove that the Cantor set $C$ is totally disconnected and 
    perfect.
\end{exercise}
\begin{pf}
    $\forany x,y\in C$, $\exist k$ such that $3^{-k}<\abs{x-y}$. 
    Thus $x,y$ must belong to two different segments in $C_k$ 
    and hence there exist some $z\not\in C$ lies between $x,y$. 
    This proves that $C$ is totally disconnected. To see that 
    $C$ is perfect, for any fix $x\in C$, note that for any 
    $\epsilon>0$, pick $k$ such that $3^{-k}<\epsilon$. Then 
    there exists a segments of $C_k$ with length $3^{-k}$ 
    containing $x$. Choose $y$ from the segment and then 
    $\abs{x-y}\leq 3^{-k}<\epsilon$. This completes the proof.
\end{pf}

\begin{exercise}\exelab{1.2}
    The Cantor set $C$ can also be described in terms of ternary 
    expressions. 
    \begin{thmenum}
        \item Every number in $[0,1]$ has a ternary expansion
        \[
            x = \sum_{k=1}^\infty a_k3^{-k},
            \quad\text{where $a_k = 0,1,2$.}
        \]
        Prove that $x\in C$ if and only if $x$ has a 
        representation as above where $a_k = 0,2$.
        \item The \textbf{Cantor-Lebesgue function} is defined 
        on $C$ by
        \[
            F(x) = \sum_{k=1}^\infty \frac{b_k}{2^k} 
            \quad\text{if $x = \sum_k a_k3^{-k}$, where 
            $b_k = \frac{1}{2}a_k$.}
        \]
        We choose the representation that $a_k=0,2$ in the 
        definition. Show that $F$ is well-defined and continuous 
        on $C$; moreover, $F(0) = 0$ and $F(1)=1$.
        \item Prove that $F:C\to[0,1]$ is surjective.
        \item $F$ can be extended to a function on $[0,1]$ as 
        follows. For $x$ in the complement of $C$, $x$ must lie 
        in some open interval $(a,b)$ with $F(a) = F(b)$. Let 
        $F(x) = F(a)$ for such $x$. Show that $F$ is continuous 
        on $[0,1]$. 
    \end{thmenum}
\end{exercise}
\begin{pf}
    For (a), we know that $C = \bigcap_k C_k$, where $C_k$ is 
    the union of closed segments. Thus in every step, we remove 
    the middle and hence all the $a_k=1$. Note that those 
    numbers with non-unique representations are the terminated 
    points and the segments removed are open sets. Furthermore, 
    such representation is unique if we restrict our choice of 
    $a_k$ to be 0 or 2 for $x\in C$. To see this, let $a_k$ be 
    the first digit that differs from $a'_k$ with $a_k>a'_k$. 
    Then we have $a_k-a'_k = 2 = \sum_{n=k+1}^\infty3^{-n}
    (a'_n-a_n) \leq 3^{-k}$, 
    which is absurd. Hence the representation must be unique.

    For (b), by the uniqueness of such representation, $F$ is 
    well-defined. Let $x_n\to x\in C$. For any 
    $\epsilon>0$, $\exist N$ such that $2^{-N+1}<\epsilon$. 
    Pick $x_n$ such that the first digit in its representation 
    that differs from $x$ occurs after $N$. Then 
    $\abs{F(x_n)-F(x)} \leq \sum_k 2^{-k-1}\abs{a^n_k-a_k} 
    \leq \sum_{k\geq N} 2^{-k} = 2^{-N+1}<\epsilon$. Hence $F$ 
    is continuous on $C$. Furthermore, $F(1) = 1$ since 
    $1=\sum_k2^{-k}\times1$ and $F(0) = 0$ since 
    $0=\sum_k2^{-k}\times0$.

    To see (c), note that every number has a binary 
    representation and thus $F$ is surjective.

    Finally, for (d), we have shown that $F$ is continuous on 
    $C$. Observe that $F$ is a piecewise function continuous at 
    each terminated points and constant functions are continuous. 
    We conclude that the extended $F$ is continuous on $[0,1]$. 
\end{pf}

\begin{exercise}\exelab{1.3}
    Consider a unit interval $[0,1]$, and let $\xi$ be a fixed 
    real number with $\xi\in(0,1)$. $C_\xi$ is defined
    iteratively as follows. $C_0 = [0,1]$. At the $n$-th step,
    $C_n$ is obtained by removing the open middle interval of
    length $\xi$ proportion from each of the closed intervals 
    in $C_{n-1}$. 
    \begin{thmenum}
        \item Prove that $C_\xi^c$ in $[0,1]$ is the union of 
        open intervals of total length 1.
        \item Show directly that $m^*(C_\xi) = 0$. 
    \end{thmenum}
\end{exercise}
\begin{pf}
    For (a), let $A_n$ be the complement of $C_n$ in $[0,1]$. 
    Then $A_n = A_{n-1}\dot{\bigcup} B_n$, where $B_n$ is the
    union of newly removed open intervals, which are of length
    $(1-\xi)^{n-1}\xi$. $A_0 = \varnothing$. Thus $m(A_n) = 
    (1-\xi)^{n-1}\xi + m(A_{n-1}) = \sum_{k\leq n}(1-\xi)^{k-1}\xi 
    \to 1$. 

    For (b), observe that $C_n\subset C_{n-1}$ and $C_n$ are 
    essentially the union of $2^n$ disjoint intervals of length 
    $(2(1-\xi))^{-n}$. Thus $m^*(C_n) = (1-\xi)^{-n}$ and then
    $m^*(C_\xi)\leq m^*(C_n) = (1-\xi)^{-n} \to 0$ as 
    $n\to\infty$. Hence $m^*(C_\xi) = 0$.
\end{pf}

\begin{exercise}\exelab{1.4}
    Construct a closed set $\hat{C}$ so that at the $k$-th step,
    one removes $2^{k-1}$ central open intervals each of length 
    $l_k$, with $l_1 + 2l_2 + \cdots + 2^{k-1}l_k < 1$ for any 
    $k$. 
    \begin{thmenum}
        \item If $l_j$ are chosen small enough, then 
        $\sum_k 2^{k-1}l_k < 1$. In this case, show that 
        $m(\hat{C}) = 1 - \sum_k 2^{k-1}l_k$.
        \item Show that if $x\in\hat{C}$, then there exists a 
        sequence of points $x_n$ such that $x_n\not\in\hat{C}$, 
        yet $x_n\to x$ and $x_n\in I_n$, where $I_n$ is a 
        subinterval in the complement of $\hat{C}$ with 
        $\abs{I_n}\to 0$.
        \item Prove that $\hat{C}$ is perfect and contains no 
        open intervals. 
        \item Show that $\hat{C}$ is uncountable.
    \end{thmenum}
\end{exercise}
\begin{pf}
    For (a), we again note that $\hat{C} = \bigcap_n C_n$, 
    where $C_n$ is the union of closed intervals with total 
    length $m(C_n) = m(C_{n-1})-2^{n-1}l_n$ and $m(C_0) 
    = [0,1]$. Also, $C_n\searrow\hat{C}$ and hence $m(\hat{C}) 
    = \lim_n m(C_n) = 1 - \sum_k 2^{k-1}l_k$.

    To prove (b), let $\epsilon>0$ be given. A subinterval in 
    $C_n$ is of length $2^{-n}(1 - \sum_{k=1}^n 2^{k-1}l_k)$. 
    Then $x_n$ can be chosen from the interval removed in 
    $n+1$-th step in the subinterval containing $x$. By our 
    construction, $x_n\not\in\hat{C}$. Since $x_n$ and $x$ lie
    in the same subinterval of $C_n$, $\abs{x_n-x}<\epsilon$ 
    for any $n\geq N$, where $N$ is chosen such that 
    $2^{-N}(1 - \sum_{k=1}^N 2^{k-1}l_k) < \epsilon$. 
    Furthermore $\abs{I_n}=l_n\to 0$ as $n\to\infty$ since 
    $\sum_{k=1}^\infty 2^{k-1}l_k<\infty$, which implies that 
    $l_k\to 0$ as $k\to\infty$.

    To prove (c), note that for any 
    $x\in\hat{C}$, we may pick $x_n\in C_n$, where $x_n$ is 
    the terminated point of the subinterval containing $x$. 
    Then $\abs{x_n-x}\leq 2^{-n}(1 - \sum_{k=1}^n 2^{k-1}l_k)
    \to 0$ as $n\to\infty$. Hence $\hat{C}$ is perfect. Also, 
    $\hat{C}$ contains no open intervals since the 
    subintervals in $C_n$ have length 
    $2^{-n}(1 - \sum_{k=1}^n 2^{k-1}l_k)$, which can be 
    arbitrarily small for large $n$.

    Finally, for (d), if $\sum_{k=1}^{\infty}2^{k-1}l_k<1$,
    since $m(\hat{C})\neq 0$, $\hat{C}$ must be uncountable.
    In general, for the case where 
    $\sum_{k=1}^{\infty}2^{k-1}l_k=1$, we may construct the 
    bijection to a fat Cantor set, which has each removed 
    intervals with same center but with length $rl_k<l_k$. 
    Such function is essentially a scalar multiplication and
    hence a bijection. But any fat Cantor set is uncountable.
    Hence so is any thin Cantor set.
\end{pf}

\begin{exercise}\exelab{1.5}
    Suppose $E$ is a given set and $O_n = \Set{x}{d(x,E)<1/n}$.
    \begin{thmenum}
        \item Show that if $E$ is compact, then $m(E)
        =\lim_{n\to\infty}m(O_n)$.
        \item Construct the counterexamples for which 
        $E$ is closed but unbounded and $E$ is bounded but open.
    \end{thmenum}
\end{exercise}
\begin{pf}
    If $E$ is compact, then $E$ is bounded, closed and hence 
    measurable. Notice that $O_n$ is open and bounded. We 
    claim that $O_n\searrow E$. Indeed, $O_n\supset O_{n+1}
    \supset E$. Also, for any $x\in O_n-E$, let $d = d(x,E)$.
    Then pick $N$ such that $1/N<d$. Then $x\not\in O_N$ and 
    hence $x\not\in O_n$. We conclude that $O_n\searrow E$.
    Thus $m(O_n)\to m(E)$ as $n\to \infty$.

    To construct the counterexamples, let 
    $E = \Set{s_n}{s_n=2\sum_{k=1}^{n}1/k,n\in\N}$. Then $E$ 
    is closed since every point in $E$ is a isolated point. 
    For any $n$, $O_n \supset (s_n-1/n,\infty]$, which has 
    an infinite measure. Thus $m(O_n)=\infty$ for all $n$.
    Hence $m(E)\neq\lim_{n\to\infty}m(O_n)$. For the second
    counterexample, let $r_k$ denote all rational numbers in 
    $(0,1)$. $E$ is the union of 
    $(r_k-2^{-k}\epsilon,r_k+2^{-k}\epsilon)$, where 
    $\epsilon\in(0,1)$. Then $E$ is bounded but open with 
    $m(E)\leq\sum_{k=1}^{\infty}2^{-k}\epsilon = \epsilon$. 
    However, since $r_k$ is dense in $(0,1)$, $O_n\supset
    (0,1)$ and hence $m(O_n)\geq 1$ for all $n$. Thus $m(E)
    \neq\lim_{n\to\infty}m(O_n)$.
\end{pf}

\begin{exercise}\exelab{1.6}
    Let $B$ be a ball in $\R^d$ of radius $r$. Prove that 
    $m(B) = v_d r^d$, where $v_d = m(B_1)$ and $B_1$ is the 
    unit ball centered at the origin with radius $1$.
\end{exercise}
\begin{pf}
    By translation and dilation, we know that $B = rB_1+b$ 
    for some $b$. Then $m(B) = r^dm(B_1) = v_d r^d$. 
\end{pf}

\begin{exercise}\exelab{1.7}
    Let $\delta\in\R^d_+$ and $E\subset\R^d$. $\delta E$ is 
    defined as $\Set{(\delta_1x_1,\ldots,\delta_dx_d)}{x\in E}$. 
    Prove that $\delta E$ is measurable if $E$ is measurable, 
    and $m(\delta E) = \delta_1\cdots\delta_dm(E)$.
\end{exercise}
\begin{pf}
    For any measurable $E\subset\R^d$, $\exist V$ open such
    that $E\subset V$ and $m(V-E)<\epsilon$. Let $V-E$ be 
    covered by a sequence of rectangles $R_n$ with 
    $\sum_n m(R_n) < \epsilon$. Furthermore, $\delta V$ is 
    also open since it is a linear transformation and hence 
    continuous. Also, $\delta V-\delta E$ is covered by 
    $\bigcup_n \delta R_n$ and $m(\delta V- \delta E)
    = m(\delta (V-E)) \leq m(\delta \bigcup_n R_n) 
    = m(\bigcup_n\delta R_n) = \sum_n m(\delta R_n) 
    = \sum_n \delta_1\cdots\delta_d m(R_n) 
    < \delta_1\cdots\delta_d\epsilon$. Since $\epsilon$ is
    arbitrarily small, $\delta E$ is measurable. Moreover, 
    if $Q_n$ is a sequence of rectangles covering $E$ with 
    $m(E)\leq\sum_n Q_n < \epsilon$, then
    $\delta Q_n$ is a sequence of rectangles covering 
    $\delta E$ with $\sum_n m(\delta Q_n) 
    = \delta_1\cdots\delta_d\sum_n m(Q_n) 
    < \delta_1\cdots\delta_d(m(E) + \epsilon)$. Since 
    $\epsilon$ is arbitrarily small, $m(\delta E) 
    = \delta_1\cdots\delta_d m(E)$. 
\end{pf}

\begin{exercise}\exelab{1.8}
    Suppose $T$ is a linear transformation in $\R^d$ and $E$
    is a measurable set in $\R^d$. Prove that $T(E)$ is 
    measurable.
\end{exercise}
\begin{pf}
    If $E$ is compact, then so is $T(E)$ since $T$ is linear 
    and hence continuous. Thus if $E$ is an $F_\sigma$ set, 
    then $T(E)$ is also an $F_\sigma$ set. This follows from 
    the fact that an $F_\sigma$ set is the countable union of
    compact sets. Next, since $T$ is linear, $\exist M$ 
    such that $\abs{T(x')-T(x)}\leq M\abs{x'-x}$ for any 
    $x,x'\in\R^d$. Then $T$ maps any cube of side length $l$
    into a cube of side length $2Ml$. Now suppose that $E$ is 
    a set of measure zero. Let $Q_n$ be a sequence of cubes 
    covering $E$ with $\sum_n m(Q_n)<\epsilon$. Then 
    $m^*(T(E))\leq\sum_n m^*(T(Q_n))\leq 2^dM^d\epsilon$ for 
    arbitrarily small $\epsilon$. Hence $T(E)$ is measurable 
    with measure zero. Finally, for any measurable set $E$,
    $\exist F$ closed such that $m(E\Delta F) = 0$. Then 
    $m(T(E)\Delta T(F)) = m(T(E\Delta F)) = 0$ and hence 
    $T(E)$ is measurable.
\end{pf}

\begin{exercise}\exelab{1.9}
    Find an open set $V$ such that $\partial \overline{V}$ has 
    positive Lebesgue measure.
\end{exercise}
\begin{pf}
    Consider the complement of a fat Cantor set in $[0,1]$ 
    obtained by letting $l_k = 4^{-k}$, denoted as $V$. 
    Being an union of open intervals, $V$ is open. Note that 
    $V$ by \exeref{1.4}, $V'$ contains the fat Cantor set. 
    Then $\partial \overline{V} = V'-V$ also contains the fat
    Cantor set and hence has positive measure.
\end{pf}

\begin{exercise}\exelab{1.10}
    Construct a decreasing sequence of continuous functions 
    $f_n\to f$ defined on $[0,1]$ while $f$ is not Riemann 
    integrable.
\end{exercise}
\begin{pf}
    Consider the removed intervals $V_n$ in each step in the 
    construction of a fat Cantor set $\hat{C}$ with 
    $l_n = 4^{-n}$. Let $I^k_n$ be the sub-intervals of $V_n$. 
    Define 
    \[
        F_n(x) = \begin{cases}
            1 & \text{$x\not\in V_n$},\\
            2\times4^n\abs{x-p^k_n} & \text{$x\in I^k_n$},
        \end{cases}
    \]
    where $p^k_n$ is the center of $I^k_n$. Then $F_n$ is 
    continuous. Put $f_n = \prod_{i=1}^{n}F_i$. Then $f_n$ is
    devreasing and continuous. Let $f_n\to f$ pointwisely. $f$
    exists since for any $x$, $f_n(x)$ is a decreasing 
    sequence bounded below. Now we claim that $f$ is 
    discontinuous on $\hat{C}$. Indeed, let $\epsilon = 1/2$ 
    be given. For any $x\in\hat{C}$, any neighborhood of $x$ 
    contains some $I^k_n$ and hence $\exist x'\in I^k_n$ such 
    that $f(x') = 0$ and thus $\abs{f(x)-f(x')} = 1>\epsilon$.
    The claim thus follows. However, by \exeref{1.4}, 
    $\hat{C}$ has positive measure and hence $f$ has positive
    measure of discontinuity. Thus $f$ is not Riemann 
    integrable.
\end{pf}

\begin{exercise}\exelab{1.11}
    Let $A$ contains all numbers in $[0,1]$ whose decimal 
    expansion contains no digit $4$. Find $m(A)$.
\end{exercise}
\begin{pf}
    Let $B_n$ consist of all numbers in $[0,1]$ in which the 
    first $4$ appears at the $n$-th decimal place. Then we 
    have $A^c = \dot{\bigcup_n} B_n$ and every $B_n$ is simply
    the union of $9^{n-1}$ disjoint open intervals of length 
    $10^{-n}$. Thus $m(A) = 1 - \sum_{n=1}^{\infty} 
    9^{n-1}10^{-n} = 1 - \frac{1}{10}\sum_{n=0}^{\infty} 
    \pth{\frac{9}{10}}^n = 8/9$.  
\end{pf}

\begin{exercise}\exelab{1.12}
    \begin{thmenum}
        \item Let $D$ be an open disc in $\R^2$ and $E$. Prove 
        that $D$ is not the disjoint union of open rectangles.
        \item Prove that an open connected set $\Omega$ is the 
        disjoint union of open rectangles if and only if $\Omega$
        is an open rectangle.
    \end{thmenum}
\end{exercise}
\begin{pf}
    (a) is proved by (b); hence we dive straight into (b). In 
    (b), the necessity is trivial. For the sufficiency, if not,
    then $\Omega$ must be composed of at least two disjoint 
    open sets. However, since $\Omega$ is connected, it cannot
    be separated into two disjoint open sets. This poses a 
    contradiction and hence $\Omega$ must be an open rectangle. 
\end{pf}

\begin{exercise}\exelab{1.13}
    \begin{thmenum}
        \item Show that a closed set is a $G_\delta$ set and 
        an open set is an $F_\sigma$ set.
        \item Give an example of an $F_\sigma$ set which is 
        not a $G_\delta$ set. 
        \item Give an example of a Borel set which is not a 
        $G_\delta$ set nor an $F_\sigma$ set.
    \end{thmenum}
\end{exercise}
\begin{pf}
    To see (a), let $F$ be a closed set. Consider $V_n 
    = \Set{x}{d(x,F)<1/n}$. Then $F = \bigcap_n V_n$ since 
    for any $x\in V_n-F$, $d(x,F)\geq \epsilon>0$ for some
    $\epsilon$. But $\exist N$ such that $1/N<\epsilon$ and 
    $x\not\in V_N$. Thus $F$ is a $G_\delta$ set. If $G$ is 
    open, $G^c$ is closed and hence a $G_\delta$ set. By 
    picking the complement of $V_n$, $G$ is an $F_\sigma$ 
    set.

    For (b), consider $\Q$. Since $\Q$ is countable, 
    $\Q = \bigcup_{q\in\Q}\{q\}$ is an $F_\sigma$ set. We now
    claim that it is not a $G_\delta$ set. If $\Q$ is a 
    $G_\delta$ set, then $\Q = \bigcap_n V_n$ for open sets 
    $V_n$. Consider $U_n = V_n-q_n$, where $q_n$ denumerates 
    all rational numbers. Then $U_n$ is dense in $\R$ for each 
    $n$. By \thmref{A.1}, $\bigcap_n U_n$ is dense in $\R$. 
    However, $\pth{\bigcap_n V_n}-\Q = \bigcap_n U_n 
    = \varnothing$, which is absurd. Hence $\Q$ is not a 
    $G_\delta$ set.

    For (c), pick $E = (\Q\bigcap\R_+)\bigcup(\Q^c\bigcap\R_-)$.
    Then $E$ is a Borel set since $\Q$ is an $F_\sigma$ set and 
    hence a Borel set. Note that $\Q^c$ is not an $F_\sigma$ 
    set since if it is, then $\Q$ is a $G_\delta$ set. But we 
    have just shown that $\Q$ is not a $G_\delta$ set. Thus 
    $\Q^c$ is not an $F_\sigma$ set. Hence $E$ is neither a 
    $G_\delta$ set nor an $F_\sigma$ set while still a Borel 
    set.
\end{pf}

\begin{exercise}\exelab{1.14}
    The \textbf{outer Jordan content} of a set $E\subset\R$ is 
    defined by
    \[
        J^*(E) = \inf\sum_{n=1}^{N}\abs{I_n},
    \]
    where the infimum is taken over all finite covering 
    rectangles $\brc{I_n}$.
    \begin{thmenum}
        \item Prove that $J^*(E) = J^*(\overline{E})$.
        \item Find a set $E$ such that $J^*(E) = 1$ while 
        $m^*(E) = 0$.
    \end{thmenum}
\end{exercise}
\begin{pf}
    For (a), it is clear that if $\brc{I_n}$ is a cover 
    consisting of rectangles of $E$, then $E\subset\overline{E}
    \subset\bigcup_{n=1}^N I_n$ as long as $I_n$ are chosen to 
    be closed. Thus by definition we can pick $\brc{I_n}$ such 
    that $J^*(E)\leq J^*(\overline{E})\leq
    \sum_{n=1}^{N}\abs{I_n} < J^*(E)+\epsilon$ 
    for arbitrarily small $\epsilon$. The first inequality 
    comes from the fact that any cover of a set is also a cover
    of any of its subset. Then it follows that $J^*(E) 
    = J^*(\overline{E})$.

    For (b), consider $E = \Q\cap[0,1]$. Then $J^*(E) = 1$ 
    since $E$ is dense in $[0,1]$ and hence any finite cover 
    of $E$ must cover $[0,1]$. Thus $J^*(E) = 1$. On the other 
    hand, $m^*(E) = 0$ since $E$ is a countable set. 
\end{pf}

\begin{exercise}\exelab{1.15}
    Define $m^*_R(E)$ by considering the infimum of volumes 
    among all countable rectangles covering $E$. Prove that 
    $m^*_R(E) = m^*(E)$ for any $E\subset\R^d$.
\end{exercise}
\begin{pf}
    Let $I_n$ be a sequence of rectangles covering $E$ with
    $\sum_n \abs{I_n} < m^*_R(E)+\epsilon$. Then by extending 
    their borders and subdividing, we obtain a sequence of 
    almost disjoint cubes $Q_k$ with same total volume. Since 
    some cubes may be discarded, we have $\sum_k\abs{Q_k}\leq
    \sum_n \abs{I_n} < m^*_R(E)+\epsilon$. Since $\epsilon$ is
    arbitrary, $m^*(E)\leq m^*_R(E)$. The reverse inequality 
    is trivial since every cube is a rectangle.  
\end{pf}

\begin{exercise}[Borel-Cantelli Lemma]\exelab{1.16}
    Let $\brc{E_k}$ be a sequence of measurable subsets of 
    $\R^d$ with $\sum_k m(E_k)<\infty$. Let 
    \[
        E = 
        \Set{x\in\R^d}{x\in E_k\text{ for infinitely many }k}
        = \limsup_k E_k.
    \]
    \begin{thmenum}
        \item Show that $E$ is measurable.
        \item Prove that $m(E) = 0$.
    \end{thmenum}
\end{exercise}
\begin{pf}
    We may write $E$ as $\bigcap_n\bigcup_{k\geq n}E_k$. For 
    (a), note that $E_k$ is measurable and hence so is their 
    countable union and countable intersection. 

    Notice that we have $E\subset\bigcup_{k\geq n}E_k$ for all
    $n$. Then $m(E)\leq m(\bigcup_{k\geq n}E_k)\leq 
    \sum_{k\geq n}m(E_k)$. Letting $n\to\infty$, we must have 
    $\sum_{k\geq n}m(E_k)\to 0$ since the series converges. It 
    follows that $m(E) = 0$.
\end{pf}

\begin{exercise}\exelab{1.17}
    Let $f_n$ be a sequence of measurable functions on $[0,1]$ 
    such that $\abs{f_n(x)}<\infty$ almost everywhere. Show 
    that $\exist$ a sequence of positive numbers $c_n$ such 
    that $\frac{f_n(x)}{c_n}\to 0$ almost everywhere.
\end{exercise}
\begin{pf}
    For any fixed $n$, pick $c_n$ such that $E_n = 
    \Set{x}{\abs{f_n(x)}/c_n > 1/n}$ has measure less than 
    $2^{-n}$. This is valid since $\abs{f_n(x)}<\infty$ 
    almost everywhere. Observe that the set of $x$ such that 
    $\abs{f_n(x)}/c_n\not\to 0$ is a subset of 
    $E = \limsup_n E_n$. Also, $\sum_n m(E_n)<\sum_n 2^{-n} = 1$.
    By \exeref{1.16}, $m(E)=0$. Thus $\abs{f_n(x)}/c_n\to 0$ 
    almost everywhere.
\end{pf}

\begin{exercise}\exelab{1.18}
    Prove that every measurable function is the limit of a 
    sequence of continuous functions almost everywhere.
\end{exercise}
\begin{pf}
    Let the measurable function be $f$ defined on $\R^d$. 
    We first consider the case where $B_n$ is the closed ball 
    with radius $n$ centered at the origin and hence has 
    finite measure. Then by Lusin's theorem, for each $n$, 
    $\exist F_{2^{-n}}$ closed such that 
    $m(B_n-F_{2^{-n}})< 2^{-n}$ and $f$ is 
    continuous on $F_{2^{-n}}$. By \thmref{A.4}, $f$ can be 
    extended to a continuous function $f_n$ on $\R^d$. Now let 
    $E_n$ denote the set of $x$ such that $f_n(x)\neq f(x)$ and 
    $E = \Set{x}{f_n(x)\not\to f}$. If $x\in E$, then 
    $x\in E_n$ for infinitely many $n$ or $f_n(x)$ would 
    eventually be equal with $f(x)$ after some large $n$.
    Observe that $E_n\subset B_n-F_{2^{-n}}$. Thus 
    $\sum_n m(E_n) < \sum_n m(B_n-F_{2^{-n}})=1$. By 
    \exeref{1.16}, $m(E) = m(\limsup_n E_n) = 0$. Thus 
    $f_n\to f$ almost everywhere.
\end{pf}

\begin{exercise}\exelab{1.19}
    \begin{thmenum}
        \item Show that if either $A$ and $B$ is open, then
        $A+B$ is open.
        \item Show that if $A$ and $B$ are closed, then $A+B$ 
        is measurable. 
        \item Show that $A+B$ might not be closed even if $A$ 
        and $B$ are closed.
    \end{thmenum}
\end{exercise}
\begin{pf}
    To prove (a), without loss of generality we may assume 
    that $A$ is open. Then for $x+y\in A+B$ where $x\in A$ 
    and $y\in B$, $\exist$ open set $V$ conatining $x$ such 
    that $V\subset A$. Then $V+y\subset A+B$ and forms a 
    neighborhood of $x+y$. Thus $A+B$ is open.

    For (b), note that any closed set is a countable union of 
    compact sets. Next, we show that the sum of two compact 
    is also compact. Let $K_1$ and $K_2$ be compact. For any 
    $x_n+y_n\in K_1+K_2$ where $x_n\in K_1$ and $y_n\in K_2$, 
    $\exist$ convergent subsequences $x_{n_k}\to x\in K_1$ and 
    thus $y_{n_{k_l}}\to y\in K_2$. Thus $x+y\in K_1+K_2$ is a 
    limit of a subsequence of $x_n+y_n$. Hence $K_1+K_2$ is 
    compact. Thus $A+B$ is measurable since it is a countable 
    union of compact sets and hence an $F_\sigma$ set. 
    
    For (c), let 
    \[
        A = \Set{\sum_{\text{odd $k<n$}}\frac{1}{k}}{n\in\N},\quad
        B = \Set{\sum_{\text{even $k<n$}}\frac{-1}{k}}{n\in\N}.
    \]
    Then $A$ and $B$ are closed since they are subsequences of 
    harmonic series or negative harmonic series. However, the 
    alternating harmonic series does converge to $\log{2}$, 
    which is irrational and hence not in $A+B$. Thus $A+B$ is 
    not closed.
\end{pf}

\begin{exercise}\exelab{1.20}
    Show that there exist closed sets $A$ and $B$ such that 
    $m(A)=m(B)=0$ while $m(A+B)>0$ in both $\R$ and $\R^2$.
\end{exercise}
\begin{pf}
    In $\R$, let $A$ be the Cantor set and $B=\frac{1}{2}A$. 
    Then $m(A)=m(B)=0$ by \exeref{1.3} and \exeref{1.7}. 
    However, for any $x\in [0,1]$ whose ternary expansion is 
    $(x_1,\ldots)$, $x_k\in\set{0,1,2}$. Now simply pick 
    $a\in A$ and $b\in B$ such that their ternary expansions 
    satisfy that if $x_k=0$, $a_k=b_k=0$; if $x_k=1$, $a_k=0$ 
    and $b_k=1$; if $x_k=2$, $a_k=2$ and $b_k=0$. Thus 
    $A+B$ contains $[0,1]$ and hence $m(A+B)>0$. 

    In $\R^2$, let $A=[0,1]\times\set{0}$ and $B=
    \set{0}\times[0,1]$. Then $m(A)=m(B)=0$ but $A+B=[0,1]^2$ 
    and hence $m(A+B)>0$.
\end{pf}

\begin{exercise}\exelab{1.21}
    Prove that there is a continuous function that maps a 
    Lebesgue measurable set to a non-measurable set. 
\end{exercise}
\begin{pf}
    Consider the Vitali set $V$ in $[0,1]$ and the Cantor set 
    $C$. By \exeref{1.2} we know that $\exist$ a surjective 
    continuous function $F:C\to [0,1]$. Then $F^{-1}(V)$ is 
    a subset of $C$ and hence measurable while $F(F^{-1}(V)) 
    = V$ is not measurable. 
\end{pf}

\begin{exercise}\exelab{1.22}
    Show that there is no function defined on $\R$ that is 
    continuous everywhere in $\R$ and also satisfies that 
    $f(x) = \chi_{[0,1]}(x)$ almost everywhere.
\end{exercise}
\begin{pf}
    Suppose such function $f$ exists. Since $f(x) = 
    \chi_{[0,1]}(x)$ almost everywhere, for any open 
    interval $I_\delta$ containing point $0$ with length 
    $\delta$, $f$ cannot differ from $\chi_{[0,1]}(x)$ on 
    $I_\delta$ since any nonempty open intervals has non-zero
    measure. Thus $f$ must take value $1$ and $0$ on 
    $I_\delta$ for every $\delta$. This contradicts to 
    the continuity of $f$ at $0$.
\end{pf}

\begin{exercise}\exelab{1.23}
    Let $f(x,y)$ be a function on $\R^2$ such that for any 
    fixed $x$, $f(x,y)$ is continuous in $y$ and vice versa. 
    Prove that $f$ is measurable on $\R^2$. 
\end{exercise}
\begin{pf}
    For any given $y$, let $f_n(x,y) = f(2^{-n}x_n,y)$ where 
    $x_n$ is the largest integer such that $2^{-n}x_n\leq x$. 
    We now claim that $f_n$ is measurable. Indeed, 
    \begin{equation*}
        \begin{split}
            \Set{(x,y)}{f_n(x,y)>a} &= \bigcup_{k\in\Z} 
            \Set{(x,y)}{2^{-n}k\leq x < 2^{-n}(k+1),f(2^{-n}k,y)>a}\\
            &= \bigcup_{k\in\Z} [2^{-n}k,2^{-n}(k+1))\times\Set{y}{f(2^{-n}k,y)>a}.
        \end{split}
    \end{equation*}
    Since $f(2^{-n}k,y)$ is continuous in $y$, $\Set{y}{f(2^{-n}k,y)>a}$ 
    is open and hence the product is measurable. Thus $f_n$ is 
    measurable. Next, we claim that $f_n\to f$ pointwisely. 
    For any $(x,y)$ and $\epsilon>0$, $\exist N$ such that 
    $\abs{2^{-n}x-x}<\delta$ for all $n\geq N$, where $\delta$ 
    is small enough such that $\abs{f(x',y)-f(x,y)}<\epsilon$ 
    when $\abs{x'-x}<\delta$. Thus 
    $\abs{f_n(x,y)-f(x,y)}<\epsilon$ for any $n\geq N$. As 
    $f_n$ is measurable and converges pointwisely to $f$, $f$ 
    is also measurable.
\end{pf}

\begin{exercise}\exelab{1.24}
    Find an enumeration $\set{r_n}$ of $\Q$ such that the 
    complement of $\bigcup_n \pth{r_n-\frac{1}{n}, r_n+\frac{1}{n}}$ 
    in $\R$ is nonempty.
\end{exercise}
\begin{pf}
    We split the index set $\N$ into $S = \Set{2^n}{n\in\N}$ 
    and $\N-S$. Next, we enumerate those rationals lying 
    outside $[0,1]$ using $\Set{a_n}{n\in S}$ and those inside 
    $[0,1]$ using $\Set{b_n}{n\in\N-S}$. Now let $r_n$ combine 
    the two enumerations. Then observe that 
    \[
        m(\bigcup_{n\in S} \pth{a_n-\frac{1}{n}, a_n+\frac{1}{n}})
        \leq \sum_{n\in S} \frac{2}{n} < \infty
    \] 
    and 
    \[
        m(\bigcup_{n\in\N-S} \pth{b_n-\frac{1}{n}, b_n+\frac{1}{n}})
        \leq m([-1,2]) < \infty.
    \]
    It follows that the complement of $\bigcup_n 
    (r_n-\frac{1}{n},r_n+\frac{1}{n})$ must have nonzero measure 
    and hence nonempty. 
\end{pf}

\begin{exercise}\exelab{1.25}
    Prove that $E$ is measurable if and only if for $\epsilon>0$, 
    $\exist$ closed set $F$ such that $m^*(E-F)<\epsilon$.
\end{exercise}
\begin{pf}
    Assume that $E$ is measurable, then so does $E^c$. Thus 
    for any $\epsilon>0$, $\exist$ open set $G$ such that 
    $m^*(G-E^c)<\epsilon$. Pick $F = G^c \subset E$, then $F$ 
    is closed and $m^*(E-F) = m^*(G-E^c)<\epsilon$. 

    Conversely, suppose that for any $\epsilon>0$, $\exist$ 
    closed set $F\subset E$ such that $m^*(E-F)<\epsilon$. 
    Then pick $G = F^c \supset E^c$, which is open. Then 
    $m^*(G-E^c) = m^*(E-F)<\epsilon$. Hence $E^c$ is 
    measurable and so does $E$. 
\end{pf}

\begin{exercise}\exelab{1.26}
    Suppose $A\subset E\subset B$ where $A$ and $B$ are measurable 
    and $m(A) = m(B)$. Prove that $E$ is measurable.
\end{exercise}
\begin{pf}
    By measurability of $B$, $\exist$ an open set $G$ such 
    that and $m^*(G-B)<\epsilon$. Then we have $m^*(G-A) = 
    m(G)-m(A) < m(B)+\epsilon - m(A) =  m(A) + \epsilon - m(A) 
    = \epsilon$, provided that $A$ is also measurable. Thus 
    $m^*(G-E)\leq m^*(G-A) < \epsilon$, which shows that $E$ 
    is measurable.
\end{pf}

\begin{exercise}\exelab{1.27}
    Suppose $E_1\subset E_2$ are compact sets in $\R^d$. Prove 
    that for any $c$ with $m(E_1)<c<m(E_2)$, $\exist$ compact 
    set $E$ such that $m(E) = c$ and $E_1\subset E\subset E_2$.
\end{exercise}
\begin{pf}
    Let $f(t) = m(E_2\bigcap D(t))$ where $D(t)=
    \Set{x}{d(x,E)\leq t}$. Then $f(0)=m(E_1)$ and $f(x)=m(E_2)$ 
    for some $x>0$. Furthermore, $f$ is continuous since for every 
    $t\in\R_+\bigcup\set{0}$, $D(t-\frac{1}{n})\nearrow D(t)$ and 
    hence $\exist N$ such that $m(D(t))-m(D(t-\frac{1}{N}))
    <\epsilon$. Thus $f(t)-f(t-\frac{1}{N}) = m(E_2\bigcap D(t))
    - m(E_2\bigcap D(t-\frac{1}{N})) \leq m(D(t)) - 
    m(D(t-\frac{1}{N}))<\epsilon$. Since $f$ is continuous and 
    $m(E_1)<c<m(E_2)$, $\exist$ $t\in [0,x]$ such that $f(t)=c$
    by intermediate value theorem. Hence $E = E_2\bigcap D(t)$ 
    is the desired compact set.
\end{pf}

\begin{exercise}\exelab{1.28}
    Let $E$ be a set with $m^*(E)>0$. Prove that for each 
    $\alpha\in (0,1)$, $\exist$ an open interval $I$ such that 
    $m^*(E\bigcap I)\geq\alpha m^*(I)$.
\end{exercise}
\begin{pf}
    Choose an open set $V$ conatining $E$ such that $m^*(E)\geq 
    \alpha m^*(V)$. Consider $I_n$ disjoint open intervals so 
    that their union is $V$. If all open intervals fail to satisfy 
    the condition, then $m^*(E) = m^*(E\bigcap V) = 
    m^*(\bigcup_n E\bigcap I_n) \leq \sum_n m^*(E\bigcap I_n) 
    < \alpha\sum_n m^*(I_n) = \alpha m^*(V)$, leading to a 
    contradiction. Thus $\exist$ an open interval $I$ such that 
    $m^*(E\bigcap I)\geq\alpha m^*(I)$.
\end{pf}

\begin{exercise}[Steinhaus Theorem]\exelab{1.29}
    Suppose $E\subset\R$ is a measurable set with $m(E)>0$. Prove 
    that the \textbf{difference set} of $E$, $\Set{x-y}{x,y\in E}$ 
    contains an open interval centered at origin.
\end{exercise}
\begin{pf}
    By \exeref{1.28}, $\exist$ an open interval $I$ such that 
    $m^*(E\bigcap I)\geq (9/10)m^*(I)$. Let $S = E\bigcap I$ and 
    suppose that $S$ does not contain an open interval centered 
    at origin. Then for a small enough $\epsilon>0$, $S$ and 
    $S+\epsilon$ are disjoint. Thus $m^*(S\bigcup(S+\epsilon))\leq 
    m^*(I\bigcup(I+\epsilon)) = m^*(I)+\epsilon$, but 
    $m^*(S\bigcup(S+\epsilon)) = 2 m^*(S) \geq 2(9/10)m^*(I)$. 
    This leads to a contradiction and hence $S$ must contain an 
    open interval centered at origin.
\end{pf}

\begin{exercise}\exelab{1.30}
    Let $E,F$ be measurable sets and $m(E),m(F)>0$. Prove that 
    $E+F$ contains an interval. 
\end{exercise}
\begin{pf}
    By \exeref{1.28}, given $\alpha,\beta\in (0,1)$, we can find 
    open intervals $I_1, I_2$ such that $m(E\bigcap I_1)\geq
    \alpha m(I_1)$ and $m(-F\bigcap I_2)\geq\beta m(I_2)$. 
    Without loss of generality, we may assume that $m(I_1)\leq 
    m(I_2)$. Then $\exist t$ such that $-t+I_1\subset I_2$. Next, 
    for any $-s\in (0,(1-\alpha)m(I_1))$, $(-t-s+I_1)\bigcap I_2$ 
    has length at least $m(I_1)+s>\alpha m(I_1)$. Now we claim 
    that $(E-t-s)\bigcap (-F)$ is nonempty. If not, then $(E-t-s)
    \bigcap(I_1-s-t)$ and $(-F)\bigcap I_2$ are disjoint. But 
    this implies that $m(((E-t-s)\bigcap(I_1-s-t))\bigcup 
    ((-F)\bigcap I_2)) = m(E\bigcap I_1)+m((-F)\bigcap I_2)
    \geq \alpha m(I_1)+\beta m(I_2)$ and also $m(((E-t-s)\bigcap
    (I_1-s-t))\bigcup ((-F)\bigcap I_2))\leq m((I_1-s-t)\bigcup 
    I_2) < m(I_2)+s$, leading to a contradiction by picking 
    $s<\alpha m(I_1)-(1-\beta)m(I_2)$ and $\alpha > 
    (1-\beta)m(I_2)/m(I_1)$. Since the intersection is nonempty, 
    so does $I = (E-t-s)\bigcap(I_1-t-s)\bigcap(-F)\bigcap I_2$. 
    For any $x\in I$, $x = e-t-s = -f$ for some $e\in E$ and 
    $f\in F$. Thus $e+f = t+s$ and $E+F$ contains an interval 
    centered at $t$ with radius $\alpha m(I_1)-(1-\beta)m(I_2)$.
\end{pf}

\begin{exercise}\exelab{1.31}
    Prove that the Vitali set $V$ is not measurable.
\end{exercise}
\begin{pf}
    Suppose $V$ is measurable. Consider translations of $V$ 
    by all rational numbers $\set{r_n}$ in $[0,1]$. We obtain 
    that $\bigcup_n V+r_n$ is a countable disjoint union of 
    measurable sets covering $[0,1]$, and hence each $V_n$ must 
    possesses a positive measure or the measure of $\bigcup_n 
    V+r_n$ must be zero. By \exeref{1.29}, the difference set 
    of $V$, denoted as $S$, contains an open interval centered 
    at origin. Note that $S$ does not contain any rational 
    numbers except $0$ by construction and hence cannot have 
    an interval having dense rational numbers in the open 
    interval, posing a contradiction. It follows that $V$ is 
    not measurable.
\end{pf}

\begin{exercise}\exelab{1.32}
    Let $V$ be the Vitali set. 
    \begin{thmenum}
        \item Prove that if $E\subset V$ is measurable, then 
        $m(E) = 0$. 
        \item Let $G\subset\R$ with $m^*(G)>0$. Prove that there 
        exists a subset of $G$ that is not measurable. 
    \end{thmenum}
\end{exercise}
\begin{pf}
    For (a), the proof is by contradiction. If $E$ has positive 
    measure, then the difference set of $E$, $S$, contains 
    an open interval centered at origin by \exeref{1.29}. But 
    $S$ does not contain any rational numbers except $0$ since 
    $E\subset V$. Then $S$ cannot contain an interval having 
    dense rational numbers in the open interval, leading to a  
    contradiction. Thus $m(E) = 0$.

    To see (b), we first note that $m^*(G)>0$ implies that 
    $G$ is uncountable since any countable set has zero outer
    measure. This implies that $G$ we can follow the 
    construction of the Vitali set to obtain a subset of $G$ 
    by considering all equivalence classes of $\sim$ where 
    $x\sim y$ if $x-y\in\Q$. This subset is again not 
    measurable by similar arguments as in \exeref{1.31}.
\end{pf}

\begin{exercise}\exelab{1.33}
    Let $V$ be the Vitali set. Show that $V^c = [0,1] - V$ 
    satisfies $m^*(V^c) = 1$ and $m^*(V) + m^*(V^c)\neq 
    m^*(V\bigcup V^c) = m^*([0,1])$ despite that $V$ and 
    $V^c$ are disjoint.
\end{exercise}
\begin{pf}
    Since $V^c\subset [0,1]$, $m^*(V^c)\leq m^*([0,1]) = 1$. 
    Suppose $m^*(V^c)<1$, then $\exist$ sequence of open 
    intervals $U_n$ whose union $U$ covers $V^c$ such that 
    $m^*(U) < 1$. Since $U$ is open, $U$ is a countable union 
    of disjoint open intervals $U'_n$ and also measurable. Then 
    $1>1-\epsilon>m(U) = \sum_n m(U'_n)$ for some $\epsilon>0$. 
    Then $\exist$ an open interval lying in $U^c$ and hence in 
    $V$, contradicting to (a) in \exeref{1.32} since an open 
    interval must have positive measure. We conclude that 
    $m^*(V^c) = 1$.

    Next, suppose that $m^*(V)+m^*(V^c)=m^*(V\bigcup V^c)=1$. 
    Then $m^*(V)=0$ and hence $V$ is measurable, which is a 
    contradiction to \exeref{1.31}. Thus $m^*(V) + m^*(V^c)
    \neq m^*(V\bigcup V^c)$.
\end{pf}

\begin{exercise}\exelab{1.34}
    Let $C_1,C_2$ be the Cantor-like sets as with constant 
    dissection ratios $\xi_1,\xi_2$ respectively. Prove that 
    $\exist F:[0,1]\to[0,1]$ such that $F$ is continuous, 
    bijective, increasing, and $F(C_1) = C_2$.  
\end{exercise}
\begin{pf}
    Note that we may write $C_1$ into $\bigcap_n C_1^n$ where 
    $C_1^n$ is the $n$th stage of the Cantor set. Then we 
    define $f$ sending any $x\in C_1$ to a sequence $a_k$ such 
    that $a_k = 0$ if $x$ lies in the left interval in the $k$th
    stage and $a_k=1$ otherwise. Then $f$ is a bijection from 
    $C_1$ to $\set{0,1}^\N$. Follow a similar construction, 
    we obtain $g$ from $\set{0,1}^\N$ to $C_2$ being another 
    bijection. $f,g$ are also continuous (by \exeref{1.2}) and 
    increasing under the order in $\set{0,1}^\N$ set to be the 
    usual order of their summation $\sum_n 2^{-n}a_n$. For the 
    rest disjoint open intervals, just map them through the 
    piecewise linear function $h$ connecting the endpoints. 
    Finally, 
    \[
        F = \begin{cases}
            g\circ f^{-1} & \text{on } C_1\\
            h & \text{on } [0,1]\bigcap(C_1)^c
        \end{cases}
    \]
    is the desired function.
\end{pf}

\begin{exercise}\exelab{1.35}
    Find a measurable function $f$ and a continuous function 
    $\Phi$ such that $f\circ\Phi$ is not measurable. 
    Furthermore, show that there exists a Lebesgue measurable 
    set that is not a Borel set.
\end{exercise}
\begin{pf}
    Let $\Phi:C_1\to C_2$ as in \exeref{1.34} with $m(C_1)>0$ 
    and $m(C_2)=0$, $N\subset C_1$ be a non-measurable set by 
    \exeref{1.32}. Take $f = \chi_{\Phi(N)}$. Then  
    \[
        (f\circ\Phi)^{-1}((0,1]) = \Phi^{-1}(\Phi(N)) = N 
    \]
    is not a measurable set. Thus $f\circ\Phi$ is not 
    measurable.Also, $\Phi(N)$ is a Lebesgue measurable set 
    since it is a subset of $C_2$ and hence has measure zero. 
    However, it is not a Borel set since $\Phi^{-1}(\Phi(N)) 
    = N$ is not measurable.
\end{pf}

\begin{exercise}\exelab{1.36}
    \begin{thmenum}
        \item Construct a measurable set $E\subset [0,1]$ such 
        that for any nonempty open subinterval $I\subset [0,1]$, 
        both $m(E\bigcap I)$ and $m(E^c\bigcap I)$ have positive
        measure. 
        \item Let $f=\chi_E$. Show that if $g(x)=f(x)$ almost 
        everywhere, then $g$ is discontinuous everywhere.
    \end{thmenum}
\end{exercise}
\begin{pf}
    First, we claim that any nonempty open interval $I$ contains 
    a pair of disjoint closed sets $S,T$ such that 
    $m(S),m(T)>0$ and $S^\circ,T^\circ = \varnothing$. Indeed, 
    For $I=(a,b)$, we can split $I$ into $(a,(a+b)/2)$ and 
    $((a+b)/2,b)$. For each of them, we can build a fat Cantor 
    set lying in the interval. Thus we have found $S,T$ as 
    required.
    
    Now considering all open intervals in $[0,1]$ with 
    rational center and radius. Since such intervals are 
    countable, we can enumerate them as $\set{I_n}$. Pick 
    a pair of disjoint closed sets with empty interiors 
    $S_1,T_1$ in $I_1$. The choices of closed disjoint 
    intervals are chosen inductively. Given $S_1,\ldots,S_n$ 
    and $T_1,\ldots,T_n$, $I_{n+1}-S_1-\cdots-S_n-T_1-\cdots
    -T_n$, since the $S_i$ and $T_i$ are closed sets with 
    empty interiors, so does their finite union. Thus we know 
    that there is an open interval in $I_{n+1}-S_1-\cdots-S_n
    -T_1-\cdots-T_n$. Choose disjoint closed sets $S_{n+1},
    T_{n+1}$ with empty interiors from the interval. Set 
    $E = \bigcup_n S_n$. Now for any open interval $I$, $I$ 
    must contain some $I_n$. Then $\exist S_{n+1}\subset 
    I_n\subset I$ is a set with positive measure and hence 
    $m(E\bigcap I)>0$. Similarly, $T_{n+1}\subset E^c\bigcap 
    I_n$ is also a set with positive measure and hence 
    $m(E^c\bigcap I)>0$.
    
    For (b), suppose that $g$ is continuous at $x$. Then for 
    every open $V\subset\R$ containing $g(x)$, $g^{-1}(V)$ is 
    open in $[0,1]$. Choosing an open interval $I\subset 
    g^{-1}(V)$, $m(E\bigcap I)>0$ and $m(E^c\bigcap I)>0$. 
    Then $f(I)=\set{0,1}$ since $f=g$ almost everywhere. This 
    means that if we pick $x$ such that $g(x)=f(x)$, then 
    $V$ contains only points $\set{0,1}$, which is absurd. 
    Thus $g$ must be discontinuous everywhere.
\end{pf}

\begin{exercise}\exelab{1.37}
    Let $\Gamma$ be the graph of a continuous function 
    $f:\R\to\R$. Show that $\Gamma$ has measure zero in $\R^2$.
\end{exercise}
\begin{pf}
    We start by considering the graph of $f$ on $[k,k+1]$, 
    $k\in\Z$. Fixing $k\in\Z$, since $f\in C([k,k+1])$, $f$ is 
    $L$-Lipschitz for some $L>0$. Consider sequences of
    rectangle covers $R_n^i$ as follows. For each $n$, let 
    \[
        R_n^i = [k+\frac{i-1}{n}, k+\frac{i}{n}]
        \times[f(k+\frac{i-1}{n})-\frac{L}{n}, 
        f(k+\frac{i-1}{n})+\frac{L}{n}],
        \quad i = 1,\ldots,n.
    \]
    Then $m(R_n^i) = \frac{2L}{n^2}$. For given $n$, 
    $R_n^i$ are almost disjoint and hence $m(\bigcup_i R_n^i)
    = \frac{2L}{n}$. Letting $n\to\infty$, we obtain that 
    the graph of $f|_{[k,k+1]}$ has measure zero. Since 
    $\Gamma$ is the countable union of the graphs of 
    $f|_{[k,k+1]}$, $k\in\Z$, $\Gamma$ has measure zero.
\end{pf}

\begin{exercise}\exelab{1.38}
    Let $a,b\geq 0$. Prove that $(a+b)^\gamma\geq 
    a^\gamma+b^\gamma$ if $\gamma\geq 1$ and $(a+b)^\gamma\leq 
    a^\gamma+b^\gamma$ if $0\leq\gamma\leq 1$. 
    ($a,b> 0$ as $\gamma=0$)
\end{exercise}
\begin{pf}
    Let $\gamma\geq 0$. Notice that $(a+t)^{\gamma-1}-
    t^{\gamma-1}\geq 0$. Then 
    \[
        0 \leq\int_{0}^{b}(a+t)^{\gamma-1}-t^{\gamma-1}\d{t} 
        = \frac{1}{\gamma}\pth{(a+t)^\gamma-t^\gamma}\at_0^b
        = \frac{1}{\gamma}((a+b)^\gamma-b^\gamma-a^\gamma).
    \]
    Rearranging the terms yields the desired result. The case 
    for reverse inequality is similar.
\end{pf}

\begin{exercise}[A-G inequality]\exelab{1.39}
    Prove that $\frac{1}{n}\sum_{i=1}^{n}x_i\geq 
    \pth{\prod_{i=1}^{n}x_i}^{1/n}$ for all $x_i\geq 0, 
    i=1,\ldots,n$.
\end{exercise}
\begin{pf}
    Since $\log$ is a concave function, by Jensen's inequality, 
    \[
        \log\pth{\frac{1}{n}\sum_{i=1}^{n}x_i} 
        \geq \frac{1}{n}\sum_{i=1}^{n}\log x_i
        = \frac{1}{n}\log\pth{\prod_{i=1}^{n}x_i}.
    \]
    Taking exponential on both sides establishes the result. 
    Note that the above proof is valid for all $x_i>0$; 
    however, the case where some of $x_i=0$ is trivial 
    since the right hand side of the inequality is $0$.
\end{pf}