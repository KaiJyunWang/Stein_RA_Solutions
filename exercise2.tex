\begin{exercise}\exelab{2.1}
    Given a collection of sets $F_1,F_2,\ldots,F_n$, construct 
    $F_1^*,F_2^*,\ldots,F_N^*$ with $N=2^n-1$ such 
    that $\bigcup_{i=1}^{N}F_i^* = \bigcup_{i=k}^{n}F_k$; 
    $F_i^*$ are disjoint; and $F_k = 
    \bigcup_{F_i^*\subset F_k}F_i^*$ for every $k$.
\end{exercise}
\begin{pf}
    Let $I$ be a combinatorics of $\set{1,\ldots,n}$. For 
    example, $I=\set{1,3,4}$ is a combinatorics. Note that 
    although the length of $I$ is undetermined, the length 
    of $I$ cannot be $0$. Now let $\set{F_i^*}$ be the 
    collection of intersections of $F_i$ for $i\in I$ and 
    $F_j^c$ for $j\not\in I$, for every given $I$. Then it 
    is clear that $F_i^*$ are disjoint, and $F_k = 
    \bigcup_{F_i^*\subset F_k}F_i^*$ since for every $x$ in 
    $F_k$, $x$ must be in some $F_i$ for $i\in I$, where $I$ 
    is over every possible combinatorics containing the one 
    induces $F_k$. 
\end{pf}

\begin{exercise}\exelab{2.2}
    Prove that if $f$ is integrable on $\R^d$ and $\delta>0$, 
    then $f(\delta x)\to f(x)$ in $L^1$ as $\delta\to 1$.
\end{exercise}
\begin{pf}
    Given $\epsilon>0$, by theorem 2.4, we know that there exists 
    continuous functions $g\in L^1$ with compact support such that 
    $\norm{f-g}<\epsilon/3$. Then $\norm{f(\delta x)-g(\delta x)}<\epsilon/3$ 
    as well. Now since $g$ is continuous on compact support, we may assume that 
    $g$ is bounded by $M$ and $\supp{g}\subset I$ where $I$ has finite positive 
    measure $S$. Then there exists $\delta$ such that $\abs{g(x)-g(\delta x)}
    <\epsilon/3MS$ for some $\delta$ sufficiently close to $1$. Hence 
    \begin{equation*}
        \norm{g(x)-g(\delta x)} = \int_{\R^d} \abs{g(x)-g(\delta x)} < \epsilon/3.
    \end{equation*}
    Then $\norm{f(x)-f(\delta x)} \leq \norm{f(x)-g(x)} + \norm{g(x)-g(\delta x)} 
    + \norm{f(\delta x)-g(\delta x)} < \epsilon$. This completes the proof.
\end{pf}

\begin{exercise}\exelab{2.3}
    Suppose $f$ is integrable on $(\pi,\pi]$ and extended to $\R$ periodically. 
    Show that 
    \begin{equation*}
        \int_{-\pi}^{\pi} f = \int_I f
    \end{equation*}
    for any interval $I$ of length $2\pi$.
\end{exercise}
\begin{pf}
    We may consider the case where $I$ is contained in two consecutive 
    intervals $I_1,I_2$ where $I_1 = (k\pi,(k+2)\pi]$ and $I_2 = 
    ((k+2)\pi,(k+4)\pi]$ for some $k\in\Z$. Then 
    \begin{equation*}
        \int_I f = \int_{I\cap I_1} f + \int_{I\cap I_2} f 
        = \int_{-(k+1)\pi + I\cap I_1} f + \int_{-(k+1)\pi + I\cap I_2} f
        = \int_{-\pi}^{\pi} f.
    \end{equation*}
    The second equality follows from the fact that the measure is invariance 
    under translation and the fact that $f$ is periodic. The third equality 
    follows from the fact that $I_1,I_2$ are disjoint and thus the length of 
    the two intervals must sum up to $2\pi$. 
\end{pf}

\begin{exercise}\exelab{2.4}
    Suppose $f$ is integrable on $[0,b]$ and 
    \begin{equation*}
        g(x) = \int_{x}^{b} \frac{f(t)}{t}dt
    \end{equation*}
    for $0<x\leq b$. Show that $g$ is integrable on $[0,b]$ and 
    \begin{equation*}
        \int_{0}^{b} g(x)dx = \int_{0}^{b} f(t)dt.
    \end{equation*}
\end{exercise}
\begin{pf}
    
\end{pf}