\begin{exercise}\exelab{2.1}
    Given a collection of sets $F_1,F_2,\ldots,F_n$, construct 
    $F_1^*,F_2^*,\ldots,F_N^*$ with $N=2^n-1$ such 
    that $\bigcup_{i=1}^{N}F_i^* = \bigcup_{i=k}^{n}F_k$; 
    $F_i^*$ are disjoint; and $F_k = 
    \bigcup_{F_i^*\subset F_k}F_i^*$ for every $k$.
\end{exercise}
\begin{pf}
    Let $I$ be a combinatorics of $\set{1,\ldots,n}$. For 
    example, $I=\set{1,3,4}$ is a combinatorics. Note that 
    although the length of $I$ is undetermined, the length 
    of $I$ cannot be $0$. Now let $\set{F_i^*}$ be the 
    collection of intersections of $F_i$ for $i\in I$ and 
    $F_j^c$ for $j\not\in I$, for every given $I$. Then it 
    is clear that $F_i^*$ are disjoint, and $F_k = 
    \bigcup_{F_i^*\subset F_k}F_i^*$ since for every $x$ in 
    $F_k$, $x$ must be in some $F_i$ for $i\in I$, where $I$ 
    is over every possible combinatorics containing the one 
    induces $F_k$. 
\end{pf}

\begin{exercise}\exelab{2.2}
    Prove that if $f$ is integrable on $\R^d$ and $\delta>0$, 
    then $f(\delta x)\to f(x)$ in $L^1$ as $\delta\to 1$.
\end{exercise}
\begin{pf}
    Given $\epsilon>0$, by theorem 2.4, we know that there exists 
    continuous functions $g\in L^1$ with compact support such that 
    $\norm{f-g}<\epsilon/3$. Then $\norm{f(\delta x)-g(\delta x)}<\epsilon/3$ 
    as well. Now since $g$ is continuous on compact support, we may assume that 
    $g$ is bounded by $M$ and $\supp{g}\subset I$ where $I$ has finite positive 
    measure $S$. Then there exists $\delta$ such that $\abs{g(x)-g(\delta x)}
    <\epsilon/3MS$ for some $\delta$ sufficiently close to $1$. Hence 
    \begin{equation*}
        \norm{g(x)-g(\delta x)} = \int_{\R^d} \abs{g(x)-g(\delta x)} < \epsilon/3.
    \end{equation*}
    Then $\norm{f(x)-f(\delta x)} \leq \norm{f(x)-g(x)} + \norm{g(x)-g(\delta x)} 
    + \norm{f(\delta x)-g(\delta x)} < \epsilon$. This completes the proof.
\end{pf}

\begin{exercise}\exelab{2.3}
    Suppose $f$ is integrable on $(\pi,\pi]$ and extended to $\R$ periodically. 
    Show that 
    \begin{equation*}
        \int_{-\pi}^{\pi} f = \int_I f
    \end{equation*}
    for any interval $I$ of length $2\pi$.
\end{exercise}
\begin{pf}
    We may consider the case where $I$ is contained in two consecutive 
    intervals $I_1,I_2$ where $I_1 = (k\pi,(k+2)\pi]$ and $I_2 = 
    ((k+2)\pi,(k+4)\pi]$ for some $k\in\Z$. Then 
    \begin{equation*}
        \int_I f = \int_{I\cap I_1} f + \int_{I\cap I_2} f 
        = \int_{-(k+1)\pi + I\cap I_1} f + \int_{-(k+1)\pi + I\cap I_2} f
        = \int_{-\pi}^{\pi} f.
    \end{equation*}
    The second equality follows from the fact that the measure is invariance 
    under translation and the fact that $f$ is periodic. The third equality 
    follows from the fact that $I_1,I_2$ are disjoint and thus the length of 
    the two intervals must sum up to $2\pi$. 
\end{pf}

\begin{exercise}\exelab{2.4}
    Suppose $f$ is integrable on $[0,b]$ and 
    \begin{equation*}
        g(x) = \int_{x}^{b} \frac{f(t)}{t}dt
    \end{equation*}
    for $0<x\leq b$. Show that $g$ is integrable on $[0,b]$ and 
    \begin{equation*}
        \int_{0}^{b} g(x)dx = \int_{0}^{b} f(t)dt.
    \end{equation*}
\end{exercise}
\begin{pf}
    Note that 
    \begin{equation*}
        \abs{g(x)} = \abs{\int_{x}^{b} \frac{f(t)}{t}dt} 
        \leq \int_{x}^{b} \abs{\frac{f(t)}{t}}dt.
    \end{equation*}
    Also, by Tonelli's theorem,
    \begin{equation*}
        \int_{0}^{b} \abs{g(x)}dx \leq \int_{0}^{b} \int_{x}^{b} \abs{\frac{f(t)}{t}}dtdx 
        = \int_{0}^{b} \int_{0}^{t} \abs{\frac{f(t)}{t}}dxdt = \int_{0}^{b} \abs{f(t)}dt 
        < \infty.
    \end{equation*}
    Hence $g$ is integrable on $[0,b]$. Now we apply Fubini's theorem to obtain 
    \begin{equation*}
        \int_{0}^{b} g(x)dx = \int_{0}^{b} \int_{x}^{b} \frac{f(t)}{t}dtdx 
        = \int_{0}^{b} \int_{0}^{t} \frac{f(t)}{t}dxdt = \int_{0}^{b} f(t)dt.
    \end{equation*}
\end{pf}

\begin{exercise}\exelab{2.5}
    Suppose $F$ is closed in $\R$ such that $m(F^c)<\infty$. Let $\delta(x) 
    = d(x, F)$. Consider 
    \begin{equation*}
        I(x) = \int_{\R} \frac{\delta(y)}{\abs{x-y}^2}dy.
    \end{equation*}
    \begin{thmenum}
        \item Prove that $\delta$ is Lipschitz and hence continuous.
        \item Show that $I(x)=\infty$ for each $x\in F^c$.
        \item Show that $I(x)<\infty$ almost everywhere in $F$.
    \end{thmenum}
\end{exercise}
\begin{pf}
    To see (a), note that for all $\epsilon>0$, there exists $z\in F$ such that 
    $\abs{x-z}\leq \delta(x)+\epsilon$, and hence 
    \begin{equation*}
        \delta(y) \leq \abs{y-z} \leq \abs{x-y} + \abs{x-z} \leq \abs{x-y} + \delta(x) + \epsilon.
    \end{equation*}
    Then 
    \begin{equation*}
        \delta(y)-\delta(x) \leq \abs{x-y} + \epsilon.
    \end{equation*}
    Since $\epsilon$ is arbitrary, we have $\delta(y)-\delta(x)\leq \abs{x-y}$. 
    Similarly, we have $\delta(x)-\delta(y)\leq \abs{x-y}$, and thus 
    $\abs{\delta(x)-\delta(y)}\leq \abs{x-y}$. This shows that $\delta$ is 
    Lipschitz with constant $1$ and hence continuous.

    For (b), let $x\in F^c$. By definition we have $\delta(y)=0$ for all 
    $y\in F$. Then 
    \begin{equation*}
        I(x) = \int_{\R} \frac{\delta(y)}{\abs{x-y}^2}dy 
        = \int_{F} \frac{\delta(y)}{\abs{x-y}^2}dy + \int_{F^c} \frac{\delta(y)}{\abs{x-y}^2}dy
        = \int_{F^c} \frac{\delta(y)}{\abs{x-y}^2}dy.
    \end{equation*}
    Next, notice that $F^c$ is open and hence we may find $B_r(x)\subset F^c$. 
    By shrinking $B_r(x)$ if necessary, we may assume that $\delta(y)>\alpha$ for 
    some positive $\alpha$ for all $y\in B_r(x)$. Then 
    \begin{equation*}
        I(x) = \int_{F^c} \frac{\delta(y)}{\abs{x-y}^2}dy 
        \geq \int_{B_r(x)} \frac{\alpha}{\abs{x-y}^2}dy 
        = \int_{0}^{r} \frac{\alpha}{y^2}dy = \infty.
    \end{equation*}
    Hence $I(x)=\infty$ for each $x\in F^c$.

    Finally, for (c), we first check that $\frac{\delta(y)}{\abs{x-y}^2}\chi_{F\times F^c}$ 
    is measurable. Indeed, it is measurable since $F, F^c$ are Borel sets and 
    thus $\chi_{F\times F^c}$ is measurable. By Tonelli's theorem, we have
    \begin{equation*}
        \begin{split}
            \int_{F} I(x)dx &= \int_{F} \int_{\R} \frac{\delta(y)}{\abs{x-y}^2}dydx \\
            &= \int_{\R} \int_{F} \frac{\delta(y)}{\abs{x-y}^2}dxdy \\
            &= \int_{F} \int_{F} \frac{\delta(y)}{\abs{x-y}^2}dxdy + \int_{F^c} \int_{F} \frac{\delta(y)}{\abs{x-y}^2}dxdy \\ 
            &= \int_{F^c} \int_{F} \frac{\delta(y)}{\abs{x-y}^2}dxdy,
        \end{split}
    \end{equation*}
    since $\delta(y)=0$ whenever $y\in F$. Now notice that $F\subset \Set{x\in\R}{\abs{x-y}\geq \delta(y)}$. 
    Then 
    \begin{equation*}
        \int_{F} \frac{1}{\abs{x-y}^2}dx \leq 2\int_{\delta(y)}^{\infty} \frac{1}{x^2}dx = \frac{2}{\delta(y)}.
    \end{equation*}
    Therefore, 
    \begin{equation*}
        \int_{F} I(x)dx = \int_{F^c} \int_{F} \frac{\delta(y)}{\abs{x-y}^2}dxdy 
        \leq \int_{F^c} \frac{2\delta(y)}{\delta(y)}dy 
        = 2m(F^c) < \infty.
    \end{equation*}
    Hence $I(x)<\infty$ almost everywhere in $F$.
\end{pf} 

\begin{exercise}\exelab{2.6}
    \begin{thmenum}
        \item Find a continuous $f:\R\to\R_+$ such that $f$ is integrable on 
        $\R$ while $\displaystyle\limsup_{x\to\infty}f(x)=\infty$. 
        \item Show that if $f$ is uniformly continuous on $\R$ and integrable, 
        then $\displaystyle\lim_{\abs{x}\to\infty}f(x)=0$.
    \end{thmenum}
\end{exercise}
\begin{pf}
    For (a), consider 
    \begin{equation*}
        f(x) = \begin{cases}
            n-n^4(n-x) & \text{if $x\in (n-1/n^3,n]$ for some $n\in\N$, $n\geq 2$}, \\
            n-n^4(x-n) & \text{if $x\in (n,n+1/n^3)$ for some $n\in\N$, $n\geq 2$}, \\
            0 & \text{otherwise}.
        \end{cases}
    \end{equation*}
    Then $f$ is continuous and integrable on $\R$ since 
    \begin{equation*}
        \int_{\R} f = \sum_{n=2}^{\infty} n\cdot \frac{1}{n^3} = \pi^2/6 - 1.
    \end{equation*}
    However, $\displaystyle\limsup_{x\to\infty}f(x)=\infty$ since $f(n)=n$ for 
    every $n\in\N$. 

    For (b), we may check the case whether $f(x)\to 0$ as $x\to\infty$ and the case 
    $x\to -\infty$ is similar. Suppose $\displaystyle\lim_{x\to\infty}f(x)\neq 0$. Then 
    there exists a sequence $x_n\to\infty$ such that $f(x_n)>2\epsilon>0$ and 
    $\abs{x_{n+1}-x_n}\geq r>0$ for some $\epsilon$ and $r$. For each $x_n$, by uniform 
    continuity, there exists $\delta_n>0$ which does not depend on $x_n$ such that 
    $\abs{f(x)-f(x_n)}<\epsilon$ for all $x\in I_n\coloneqq (x_n-\delta, x_n+\delta)$. 
    Then $f(x)\geq \epsilon$ if $x\in I_n$ for some $n\in\N$. Note lastly that by 
    further shrinking $\delta<r/2$ if necessary, we may assume that $I_n$ are disjoint. 
    Then 
    \begin{equation*}
        \int_{\R} f \geq \int_{\bigcup_{n\in\N} I_n} f = \sum_{n=1}^{\infty} \int_{I_n} f 
        \geq \sum_{n=1}^{\infty} \epsilon\cdot 2\delta = \infty.
    \end{equation*}
    This contradicts the fact that $f$ is integrable on $\R$. Hence $\displaystyle
    \lim_{x\to\infty}f(x)=0$. By similar argument, we have $\displaystyle
    \lim_{x\to-\infty}f(x)=0$. This completes the proof.
\end{pf}

\begin{exercise}\exelab{2.7}
    Suppose $f:\R^d\to\R$ is measurable on $\R^d$. Let $\Gamma = \Set{(x,y)\in\R^d\times\R}
    {y=f(x)}$ be the graph of $f$. Show that $\Gamma$ is measurable in $\R^{d+1}$ and $m(\Gamma)=0$. 
\end{exercise}
\begin{pf}
    Note that it suffices to show that $\Gamma\cap (B\times [n,n+1])$ is measure zero for all 
    $n\in\Z$ and $B\subset\R^d$ is a box since $\Gamma$ is a countable union of such sets. 
    Then we know that by Lusin's theorem, for any $\epsilon>0$, there exists a closed 
    set $F_\epsilon\subset B$ such that $m(B-F_\epsilon)\leq\epsilon$ and $f$ is continuous 
    on $F_\epsilon$. Then by similar arguments in \exeref{1.37}, we can conclude that 
    $\Gamma\cap (F_\epsilon\times[n,n+1])$ is measure zero. On the other hand, 
    $(B-F_\epsilon)\times[n,n+1]$ has measure at most $\epsilon$. Combining these two facts, 
    we conclude that $m(\Gamma\cap (B\times [n,n+1]))\leq\epsilon$. Since $\epsilon$ 
    is arbitrary, we have $m(\Gamma\cap (B\times [n,n+1]))=0$. This completes the 
    proof.
\end{pf}

\begin{exercise}\exelab{2.8}
    If $f$ is integrable on $\R$, show that $F(x) = \int_{-\infty}^{x} f(t)dt$ is uniformly 
    continuous. 
\end{exercise}
\begin{pf}
    Let $\epsilon>0$ be given. Since the set of continuous functions with compact support 
    is dense in $L^1$, there exists a continuous function $g$ with compact support such that 
    $\norm{f-g}<\epsilon/2$. Also, since $g$ is continuous on compact support, we may assume 
    that $g$ is bounded by $M$ on $\R$. Then if $\abs{x-y}<\delta = \frac{\epsilon}{2M}$, say 
    $x\leq y$, we have
    \begin{equation*}
        \begin{split}
            \abs{F(x)-F(y)} 
            &= \abs{\int_{-\infty}^{x} f(t)dt - \int_{-\infty}^{y} f(t)dt} \\
            &\leq \int_{y}^{x} \abs{f(t)}dt \\
            &\leq \int_{y}^{x} \abs{f(t)-g(t)}dt + \int_{y}^{x} \abs{g(t)}dt \\
            &\leq \epsilon/2 + \int_{y}^{x} Mdt \\ 
            &= \epsilon/2 + M\abs{x-y} < \epsilon.
        \end{split}
    \end{equation*}
    Hence $F$ is uniformly continuous.
\end{pf}

\begin{exercise}[Tchebyshev's inequality]\exelab{2.9}
    Suppose $f$ is integrable on $\R$ and $f\geq 0$. If $\alpha>0$ and 
    $E_\alpha = f^{-1}(\set{(\alpha,\infty]})$, show that
    \begin{equation*}
        m(E_\alpha) \leq \frac{1}{\alpha}\int_{\R} f.
    \end{equation*}
\end{exercise}
\begin{pf}
    \begin{equation*}
        \int_{\R} f \geq \int_{E_\alpha} f \geq \int_{E_\alpha} \alpha = \alpha m(E_\alpha).
    \end{equation*}
    Rearranging the inequality gives the desired result.
\end{pf}