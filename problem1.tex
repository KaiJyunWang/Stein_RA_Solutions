\begin{problem}\prolab{1.1}
    Given an irrational $x$, show that there exists infinitely 
    many fractions $\frac{p}{q}$ where $p,q$ are coprime and 
    $\abs{x-\frac{p}{q}}\leq\frac{1}{q^2}$. In spite of this, 
    prove that the set of those $x\in\R$ such that ther exist 
    infinitely many fractions $\frac{p}{q}$, where $p,q$ are 
    coprime and $\abs{x-\frac{p}{q}}\leq
    \frac{1}{q^{2+\epsilon}}$ for some small $\epsilon>0$, 
    has measure zero.
\end{problem}
\begin{pf}
    Let $x$ be irrational. For $m=1,\ldots,N+1$, choose $n_m$ 
    to be the greatest integer such that $n_m\leq mx$. Then 
    $mx-n_m\in(0,1)$ since $x$ is irrational. Next, devide 
    $(0,1)$ into $N$ intervals, each of length $1/N$. Since 
    our choice of $m$ produces $N+1$ $mx-n_m\in(0,1)$, by 
    pigeon-hole principle, there is an interval containing 
    two numbers of the form $mx-n_m$. Then 
    \[
        \abs{(m-m')x-(n_m-n_{m'})} 
        = \abs{mx-n_m-(m'x-n_{m'})} 
        \leq \frac{1}{N}.
    \] 
    Without loss of generality, we may assume that $m>m'$. 
    Set $q = m-m'$ and $p = n_m-n_{m'}$. Then $1\leq q\leq N$ 
    and $\abs{x-\frac{p}{q}}\leq\frac{1}{qN}\leq\frac{1}{q^2}$. 
    Our choice of $N$ is arbitrary and hence there are 
    infinitely many such fractions. 

    To see the second part, for each $q\in\N$, let 
    \[
        E_q = \bigcup_{p\in\Z, (p,q)=1} 
        \Set{x}{\abs{x-\frac{p}{q}}<\frac{1}{q^{2+\epsilon}}}. 
    \]
    Then each $E_q$ is a union of intervals of length 
    $\frac{2}{q^{2+\epsilon}}$. Note that for every given $q$, 
    there are at most $2q$ integers $p$ such that $(p,q)=1$. 
    Thus $m(E_q)\leq \frac{4q}{q^{2+\epsilon}} 
    = \frac{4}{q^{1+\epsilon}}$. Then $\sum_q m(E_q) < \infty$. 
    The set of $x$ such that there exist infinitely many 
    fractions $\frac{p}{q}$, where $p,q$ are coprime is 
    $\limsup_q E_q$ and hence has measure zero by 
    \exeref{1.16}.
\end{pf}

\begin{problem}\prolab{1.2}
    Prove that every open set $\Omega\subset\R^d$ can be 
    written as an union of almost disjoint closed cubes 
    $Q_i$ with the property: $\exist c_1,c_2>0$ such that 
    $c_1 l(Q_i)\leq d(Q_i,\Omega^c)\leq c_2 l(Q_i)$ for all 
    $i$, where $l(Q_i)$ is the side length of $Q_i$.
\end{problem}
\begin{pf}
    We find the cubes through the algorithm as follows. 
    First, consider the closed cubes generated by lattice 
    $\Z^d$. Collect those cubes lying in $\Omega$ and denote 
    them as $Q_1^k$. For the rest of cubes which intersect 
    $\Omega$, partition them into cubes of half the original 
    side length. Collect those cubes lying in $\Omega$ and 
    denote them as $Q_2^k$. Repeat the procedure. We 
    obtain a sequence of collections of cubes $Q_n^k$. 
    This sequence of cubes is clearly almost disjoint. This 
    sequence of cubes also covers $\Omega$ since for any 
    $x\in\Omega$, $\exist N$ such that $B_{2^{-N}}(x)\subset 
    \Omega$. Then $x$ is contained in some cube in 
    $\set{Q_n^k}_{n=1}^{N+1}$. $\set{Q_n^k}$ forms a cover of 
    $\Omega$. 

    Next, we claim that $d(Q_n^k,\Omega^c)/l(Q_n^k)$ is 
    bounded above. If $d(Q_n^k,\Omega^c)>\sqrt{d}l(Q_n^k)$, 
    then $Q_n^k$ is contained in a larger cube within $\Omega$. 
    This contradicts the construction of $Q_n^k$. Also, 
    $d(Q_n^k,\Omega^c)/l(Q_n^k)$ is bounded below. This is 
    because $d(Q_n^k,\Omega^c)>\epsilon>0$ for some $\epsilon$ 
    since $Q_n^k$ is a compact set contained in $\Omega$ and 
    $\Omega^c$ is an open set. Thus we have $d(Q_n^k,\Omega^c)
    >\epsilon 2^{-i+1} = \epsilon l(Q_n^k)>0$ for some 
    $\epsilon>0$. This furnishes the proof.   
\end{pf}

\begin{problem}\prolab{1.3}
    Find an example of a measurable set $C\subset[0,1]$ such 
    that $m(C)=0$, yet the difference set of $C$ contains a 
    non-trivial interval centered at origin.
\end{problem}
\begin{pf}
    Let $C$ be the standard Cantor set. Then $m(C)=0$. We 
    shall prove that the difference set of $C$ is exactly 
    $[-1,1]$. First, write $C=\Set{\sum_k 3^{-k}a_k}
    {a_k\in\set{0,2}}$ by \exeref{1.2}. Then the difference 
    set of $C$ is $S = \Set{\sum_k 3^{-k}c_k}
    {c_k\in\set{-2,0,2}}$. Given $x\in[-1,1]$, define 
    $s_0 = 0$ and 
    \[
        s_{n+1} = \begin{cases*}
            s_n+\frac{2}{3^n} & \text{if $s_n+\frac{2}{3^n}$ 
            is closer to $x$ comparing to the other two,}\\ 
            s_n & \text{if $s_n$ is closer to $x$ comparing 
            to the other two,}\\ 
            s_n-\frac{2}{3^n} & \text{if $s_n-\frac{2}{3^n}$ 
            is closer to $x$ comparing to the other two.}
        \end{cases*}
    \]
    It is clear that $\abs{s_n-x}\leq 3^{-n}$ and hence 
    $s_n\to x$. Thus $x\in S$. This shows that $[-1,1]\subset 
    S$. The converse is trivial. Thus $S = [-1,1]$. As a 
    immediate consequence, $S$ contains a non-trivial 
    interval centered at origin.
\end{pf}

\begin{problem}\prolab{1.4}
    Let $E\subset\R^d$ be a rectangle and $f:E\to\R$ be a 
    function bounded by $M$. Prove that $f$ is Riemann 
    integrable if and only if the set of discontinuities of 
    $f$ has measure zero. \footnote{In this problem, I prove 
    a generalized version.}
\end{problem}
\begin{pf} 
    Define the oscillation of $f$ at $x$ as 
    \[
        \osc{f}{x} = \inf_{U:x\in U} \diam{f(U)},
    \]
    where $U$ is open. 
    
    We first claim that $f$ is continuous at $x$ if and only 
    if $\osc{f}{x}=0$. Indeed, if $f$ is continuous at $x$, 
    then $\forany\epsilon>0$, $\exist\delta>0$ such that 
    $\abs{f(x)-f(y)}<\epsilon$ for all $y\in B_\delta(x)$. 
    Then $\diam{f(B_\delta(x))}\leq 2\epsilon$. Since 
    $\epsilon$ is arbitrary, $\osc{f}{x}=0$. Conversely, if 
    $\osc{f}{x}=0$, then $\forany\epsilon>0$, $\exist$ open 
    $U$ containing $x$ such that $\diam{f(U)}<\epsilon$. 
    This implies that $\abs{f(x)-f(y)}<\epsilon$ for all 
    $y\in U$ and hence $f$ is continuous at $x$. 

    Next, let $D_\epsilon$ collect all points $x$ such that 
    $\osc{f}{x}\geq\epsilon>0$. We claim that $D_\epsilon$ 
    is closed. For any convergent sequence $x_k\in D_\epsilon$, 
    let $x_k\to x$. For any open $U$ containing $x$, $\exist N$ 
    such that $x_k\in U$ for all $k\geq N$. Then $\exist$ an 
    open neighborhood of $x_N$, $U'$, such that $U'\subset U$ 
    and $\diam{f(U')}\geq\epsilon$. Hence $\osc{f}{x}\geq
    \epsilon$ and $x\in D_\epsilon$, showing that $D_\epsilon$ 
    is closed. Observe that $D = \bigcup_{n=1}^{\infty}D_{1/n}$.
    
    Now suppose that $f$ is Riemann integrable. Then for any 
    $\epsilon>0$, $\exist\mathcal{P}$ such that $\mathrm{U}
    (f,\mathcal{P})-\mathrm{L}(f,\mathcal{P})<\frac{1}{n}$ and 
    $\norm{\mathcal{P}}<\frac{1}{n}$. Then 
    \begin{equation*}
        \begin{split}
            &\quad\sum_{\substack{Q\in\mathcal{P},\\ Q\bigcap 
            D_\frac{1}{n}\neq\varnothing}} (\sup_Q f -\inf_Q f)\abs{Q} 
            + \sum_{\substack{Q\in\mathcal{P}, \\ Q\bigcap 
            D_\frac{1}{n} = \varnothing}} (\sup_Q f -\inf_Q f)\abs{Q}\\
            &= \sum_{Q\in\mathcal{P}}(\sup_Q f -\inf_Q f)\abs{Q} 
            = \mathrm{U}(f,\mathcal{P})-\mathrm{L}(f,\mathcal{P}) 
            < \epsilon.
        \end{split}
    \end{equation*}
    Note that $\sup_Q f -\inf_Q f = \diam{f(Q)}$. This gives 
    that $2Mm^*(D_\frac{1}{n})<\epsilon$ for every $n$. Since 
    $\epsilon$ is arbitrary, we conclude that $m^*(D_\frac{1}{n}) 
    = 0$ for each $n$. Thus $D$ is an union of sets of measure 
    zero and hence also has measure zero. 

    For the converse, suppose that $m(D)=0$. Then $D_\epsilon$ 
    also has measure zero. Let $\mathcal{P}$ be a partition 
    on $E$ with $\norm{\mathcal{P}}<\delta$ for some 
    $\delta>0$, which will be determined later. Then 
    \begin{equation*}
        \begin{split}
            \mathrm{U}(f,\mathcal{P})-\mathrm{L}(f,\mathcal{P}) 
            &= \sum_{Q\in\mathcal{P}}(\sup_Q f -\inf_Q f)\abs{Q}\\ 
            &= \sum_{\substack{Q\in\mathcal{P},\\ Q\bigcap 
            D_\epsilon = \varnothing}}(\sup_Q f -\inf_Q f)\abs{Q} 
            + \sum_{\substack{Q\in\mathcal{P},\\ Q\bigcap 
            D_\epsilon\neq\varnothing}}(\sup_Q f -\inf_Q f)\abs{Q}
        \end{split}
    \end{equation*}
    For the first term, $\sup_Q f -\inf_Q f < \epsilon$ for 
    $\norm{\mathcal{P}}<\delta_1$ for some $\delta_1>0$. And 
    thus the first term is bounded by $\epsilon m(E)$. For the 
    second term, $\sup_Q f -\inf_Q f < 2M$ and since 
    $D_\epsilon$ has measure zero, $\exist Q_k$ cubic cover 
    of $D_\epsilon$ such that $\sum_k \abs{Q_k}<\epsilon$. 
    Now if $\diam{Q}<\delta_2$ for some $\delta_2>0$, then 
    those $Q$ intersecting $D_\epsilon$ nonempty are subset 
    of $\bigcup_k Q_k$. Thus the second term is bounded by 
    $2M\epsilon$. Choosing $\delta = \min\set{\delta_1, 
    \delta_2}$ yields that 
    \[
        \mathrm{U}(f,\mathcal{P})-\mathrm{L}(f,\mathcal{P}) 
        < \epsilon m(E) + 2M\epsilon
    \]
    whenever $\norm{\mathcal{P}}<\delta$. Since $\epsilon$ is 
    arbitrary, $f$ is Riemann integrable.
\end{pf}

\begin{problem}\prolab{1.5}
    Suppose $E$ is measurable with $m(E)<\infty$, $E=E_1\bigcup 
    E_2$ and $E_1\bigcap E_2=\varnothing$. Prove that if 
    $m^*(E_1) + m^*(E_2) = m(E)$, then $E_1$ and $E_2$ are 
    measurable. 
\end{problem}
\begin{pf}
    Suppose not. Then $\exist \epsilon>0$ such that 
    $m^*(V_1-E_1)>\epsilon$ and $m^*(V_2-E_2)>\epsilon$ for 
    any open sets $V_1,V_2$ covering $E_1$ and $E_2$ 
    respectively. Then $m(V_1\bigcup V_2-E) = m(V_1\bigcup V_2) 
    - m(E) = m(V_1)-m^*(E_1)+m(V_2)-m^*(E_2) = m^*(V_1-E_1) + 
    m^*(V_2-E_2)>2\epsilon$ by assumption. It remains to show 
    that for every open $V$ covering $E$, $V\supset V_1
    \bigcup V_2$ for some open $V_1,V_2$ covering $E_1,E_2$ 
    respectively. Indeed, one may simply consider $V_1 = 
    \Set{x}{d(x,E_1)<1/n}\bigcap V$ and $V_2 = 
    \Set{x}{d(x,E_2)<1/n}\bigcap V$ for some $n\in\N$. 
    Thus we obtain a contradiction since $E$ is measurable.
    We conclude that $E_1$ and $E_2$ are measurable.
\end{pf}

\begin{definition*}
    A set $E$ is said to be \textbf{well-ordered} with respect 
    to a binary relation $\leq$ if for every nonempty subset 
    $A$ of $E$, $\exist a\in A$ such that $a\leq b$ for all 
    $b\in A$.
\end{definition*}

\begin{definition*}
    Given a nonempty partially ordered set $\pth{E,\leq}$, 
    $A\subset E$ is a \textbf{maximal linearly ordered subset} 
    of $E$ if for any $B$ such that $A\subset B\subset E$, 
    $B$ is not linearly ordered.
\end{definition*}

\begin{definition*}
    If $\F$ is a family of sets and $\mathcal{C}\subset\F$, we 
    call $\mathcal{C}$ a \textbf{subchain} of $\F$ if for any 
    $A,B\in\mathcal{C}$, $A\subset B$ or $B\subset A$.
\end{definition*}

\begin{problem}\prolab{1.6}
    Prove that the following are equivalent. 
    \begin{thmenum}
        \item \textbf{Axiom of choice}: $\exist$ a choice function 
        $f:\mathcal{P}(X)\setminus\set{\varnothing}\to A$, with 
        $A\mapsto f(A)\in A$ for any $A\in\mathcal{P}(X)\setminus
        \varnothing$.
        \item \textbf{Well-ordering principle}: Every set can be 
        well-ordered.
        \item \textbf{Hausdorff maximal principle}: Every nonempty 
        partially ordered set has a maximal linearly ordered 
        subset.
        \item \textbf{Zorn's lemma}: Let $P$ be partially ordered. 
        If every linearly ordered subset of $P$ has an upper 
        bound, then $P$ has a maximal element.  
    \end{thmenum}
\end{problem}
\begin{pf}
    We start by proving that (a)$\Rightarrow$(c)\footnote{The 
    proof is adapted from the appendix in Rudin's \textit{Real 
    and Complex Analysis}.}. We begin with the following 
    claim. 

    \textbf{Claim}: Suppose $\F\subset\mathcal{P}(X)$ is a 
    nonempty collection of sets such that the union of every 
    subchain of $\F$ belongs to $\F$. Suppose $g$ is a 
    function which associates to each $A\in\F$ a set 
    $g(A)\in\F$ such that $A\subset g(A)$ and $g(A)-A$ 
    consists of at most one element. Then $\exist A\in\F$ 
    such that $g(A)=A$.
    
    Fix $A_0\in\F$. Call a subcollection $\F'$ of $\F$ a 
    \textbf{tower} if (i) $A_0\in\F'$, (ii) the union of every 
    subchain of $\F'$ belongs to $\F'$, and (iii) $g(A)\in\F'$ 
    for any $A\in\F'$. Then it is clear that the family of all 
    towers is nonempty for if $\F_1$ is the collection of all 
    $A\in\F$ such that $A_0\subset A$, then $\F_1$ is a tower. 
    Let $\F_0$ be the intersection of all towers. Then $\F_0$ 
    is a minimal tower in the sense that any proper 
    subcollection of $\F_0$ is not a tower. Also, $A_0\subset 
    A$ for any $A\in\F_0$. We now show that $\F_0$ is a 
    subchain of $\F$. 

    Let $\Gamma$ collect all $C\in\F_0$ such that for every 
    $A\in\F_0$, $A\subset C$ or $C\subset A$. For each $C\in 
    \Gamma$, let $\Phi(C)$ be the collection of all $A\in\F_0$ 
    such that either $A\subset C$ or $g(C)\subset A$. Fix $C 
    \in\Gamma$ and $A\in\Phi(C)$. It is clear that (i) and (ii) 
    hold for $\Gamma$ and $\Phi(C)$. It remains to show that 
    $g(A)\in\Phi(C)$. If $A\in\Phi(C)$, then there are three 
    cases: $A\subsetneq C$, $A=C$, or $g(C)\subset A$. In the 
    case $A\subsetneq C$, $C$ cannot be a proper subset of 
    $g(A)$, otherwise $g(A)-A$ would contain at least two 
    elements; since $C\in\Gamma$, $g(A)\subset C$. In the case 
    $A=C$, $g(A)=g(C)\in\Phi(C)$. In the case $g(C)\subset A$, 
    $g(C)\subset A\subset g(A)$ and hence $g(A)\in\Phi(C)$. 
    Thus $\Phi(C)$ is a tower. Then the minimality of $\F_0$ 
    implies that $\Phi(C)=\F_0$ for every $C\in\Gamma$. Now if 
    $A\in\F_0$ and $C\in\Gamma$, then either $A\subset C$ or 
    $g(C)\subset A$. This implies that $g(C)\in\Gamma$. Thus 
    $\Gamma$ is a tower and the minimality of $\F_0$ again 
    implies that $\Gamma=\F_0$. We now see that $\F_0$ is 
    linearly ordered by the definition of $\Gamma$ and thus 
    a subchain of $\F$. 

    Lastly, to show that $\exist A\in\F$ such that $g(A)=A$, 
    let $A$ be the union of $\F_0$. By (ii) and (iii), $A\in
    \F_0$ and $g(A)\in\F_0$. Since $A$ is the largest member 
    in $\F_0$ and $A\subset g(A)$, we have $A=g(A)$. The claim 
    is furnished. 

    Now we proceed to prove that (a)$\Rightarrow$(c). Let $\F$ 
    be the collection of all linearly ordered subsets of the 
    partially ordered set $P$. Since every subset of $P$ 
    consisting of exactly one element is linearly ordered, $\F$ 
    is nonempty. Let $g$ be a choice function on $\F$. Note 
    that the union of any chain of linearly ordered sets is 
    linearly ordered. Now let $f$ be a choice function for $P$. 
    If $A\in\F$, let $A^*$ be the set of all x in the 
    complement of $A$ such that $A\bigcup\set{x}\in\F$. If $A^* 
    \neq\varnothing$, put $g(A)=A\bigcup\set{f(A^*)}$. If 
    $A^*=\varnothing$, put $g(A)=A$. By the claim, $A^*=
    \varnothing$ for at least one $A\in\F$, and such $A$ is the 
    desired maximal linearly ordered subset. 

    Next, (c)$\Rightarrow$(d) since an upper bound of a 
    linearly ordered subset is a maximal element of $P$. 

    For (d)$\Rightarrow$(b)\footnote{The proof is from 
    Folland's \textit{Real Analysis: Modern Techniques and 
    Their Applications}.}, let $E$ be a set and $\A$ collect 
    all well-orderings of subsets of $E$. We define a partial 
    order $\prec$ on $\A$: If $\leq_1$ and $\leq_2$ are 
    well-orderings on the subsets $E_1$ and $E_2$, then 
    $\leq_1\prec\leq_2$ if (i) $E_1\subset E_2$ and $\leq_1, 
    \leq_2$ agree on $E_1$, and (ii) if $x\in E_2-E_1$, then 
    $y\leq_2 x$ for all $y\in E_1$. Now every linearly ordered 
    subcollect of $\A$ has an upper bound by taking the union 
    of the subcollection. Applying Zorn's lemma, we obtain a 
    maximal element $\leq$ in $\A$. This maximal element must 
    be a well-ordering on $E$ since if $\leq$ is only a 
    well-ordering on $C\subsetneq E$, and $x\in E-C$, then 
    $\leq$ can be extended to a well-ordering on $C\bigcup
    \set{x}$, contradicting the maximality of $\leq$.   

    Finally, to prove that (b)$\Rightarrow$(a), let $X$ be a 
    set, $A\subset X$, and define $f(A)$ be the unique minimal 
    element of $A$ if $A\neq\varnothing$. $f$ is the desired 
    choice function.
\end{pf}

\begin{problem}\prolab{1.7}
    Let $f\in C^2([0,1])$ and $\Gamma\subset\R^2$ be the graph 
    of $f$ on $[0,1]$. Prove that the following are equivalent. 
    \begin{thmenum}
        \item $m(\Gamma + \Gamma) > 0$. 
        \item $\Gamma + \Gamma$ contains a nonempty open set. 
        \item $f$ is not linear.
    \end{thmenum}
\end{problem}
\begin{pf}
    (b)$\Rightarrow$(a) is trivial since every nonempty open 
    set has positive measure. 

    (a)$\Rightarrow$(c): Suppose $f$ is linear. Then $f(x) = 
    ax+b$ for some $a,b\in\R$. Then
    \begin{equation*}
        \begin{split}
            \Gamma+\Gamma 
            &= \Set{(x+y,f(x)+f(y))}{x,y\in[0,1]}\\ 
            &= \Set{(x+y,a(x+y)+2b)}{x,y\in[0,1]}\\ 
            &= \Set{(z,az+2b)}{z\in[0,2]}
        \end{split}
    \end{equation*}
    is measure zero 
    since it is a line segment in $\R^2$. 

    (c)$\Rightarrow$(b): Suppose $f$ is not linear. Then we 
    may assume that $f''\neq 0$ has same sign on $(a,b)
    \subset[0,1]$. Define $\phi(x,y) = (x+y,f(x)+f(y))$. Then 
    \[
        D\phi = \begin{pmatrix}
            1 & 1\\ 
            f'(x) & f'(y)
        \end{pmatrix}
    \]
    is invertible if $x\neq y$. By the inverse function 
    theorem, $\exist$ open sets $U,V$ such that $\phi(U) = V$ 
    is invertible, $U\subset(a,b)^2-\Set{(x,y)}{x=y}$ and $V$ 
    is open in $\R^2$. Thus $\Gamma+\Gamma$ contains a nonempty 
    open set $V$.
\end{pf}

\begin{definition*}
    Two sets $A,B\subset\R^d$ are said to be \textbf{similar}, 
    denoted by $A\sim B$, if $\exist \delta>0$ and $a\in\R^d$ 
    such that $A = a+\delta B$.
\end{definition*}

\begin{problem}\prolab{1.8}
    Suppose $A,B$ are open sets of finite positive measure in 
    $\R^d$. Then the equality in Brunn-Minkowski inequality 
    holds if and only if $A,B$ are convex and $A\sim B$. 
\end{problem}
\begin{pf}
    Suppose $A,B$ are convex and $A\sim B$. Since $A,B$ are 
    open, $A+B$ is open and also measurable. By the assumption, 
    $\exist \delta>0$ and $a\in\R^d$ such that $A = 
    a+\delta B$. Then $m(A+B) = m((a+\delta B)+B) = 
    m((1+\delta)B) = (1+\delta)^d m(B)$ by the convexity. 
    Thus $m(A+B)^{1/d} = (1+\delta)m(B)^{1/d} = m(B)^{1/d} + 
    m(a+\delta B)^{1/d} = m(B)^{1/d} + m(A)^{1/d}$. 

    The converse is actually false. One may simply take $A=B
    =(-1,0)\bigcup(0,1)$. Then $m(A+B)=m((-1,1)+(-1,1))= 
    m((-2,2)) = 4 = 2+2 = m(A)+m(B)$. However, $A,B$ are not 
    convex.
\end{pf}
